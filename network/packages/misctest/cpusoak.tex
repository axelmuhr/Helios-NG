\documentstyle{article}
\title{Helios CPU Soak Program}
\author{Bart Veer \\ PSL-HEL-BLV-93-004.1}
\begin{document}
\maketitle

The purpose of this program is to soak up CPU time and hence test
the behaviour of Helios under stress conditions. There are
a variety of options for starting up multiple threads, multiple tasks,
and so on.

\begin{description}
\item [-t no] Start up the specified number of threads all of which
attempt to soak up CPU time. By default the program will only run a
single thread.
\item [-T no] Start up the specified number of tasks, each of which
will run the appropriate number of threads. By default the program
will not start up any additional threads.
\item [-cdl] When starting up multiple tasks the program will normally
clear the CDL flag first, to ensure that all the load occurs on the same
processor. The {\tt remote} command should be used to run the program
on specific processors. If the {\tt -cdl} option is given then the
CDL flag is set and the tasks will be distributed automatically over
the network.
\item [-i] Perform integer arithmetic only to soak up the available
CPU time. This is the default behaviour.
\item [-f] Perform floating point arithmetic only.
\item [-m] Perform continuous memory allocation and deallocation.
Note that having multiple threads all running with this option is
not very sensible as they will lock each other out.
\item [-s] Start up pairs of threads which continuously signal each
other and wait on semaphores. 
\item [-r] For each thread choose one of the above run modes at
random.
\item [-d] Detach, this causes the program to run forever.
\item [-x] Exit cleanly. By default the program just calls {\tt exit()}
and leaves it to the system to clean up all the threads. With this
option the program will actually wait for all the threads to finish
first.
\item [-p pri] Run at the specified priority. The number should be
the same as the normal Helios priorities.

\begin{description}
\item [-32768] HighestPri (not recommended)
\item [-24576] DevicePri (also not recommended)
\item [-16384] HighServerPri (not a very good idea)
\item [-8192] ServerPri (living dangerously)
\item [0] StandardPri (the default)
\item [8192] BackgroundPri
\item [16384] LowBackgroundPri
\item [32767] IdelPri
\end{description}

\item [time] The desired run time in days, hours and minutes. By
default the program will run until the user hits a key. If the program is
to be run in the background then the user should either specify a run time
or use the detach option.
\end{description}

Please note that there is a delay of ten seconds before the threads or
tasks will actually start soaking up CPU time. This is to ensure that the
system is fully initialised first.

\end{document}
