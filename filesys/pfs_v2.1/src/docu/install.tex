\chapter{Installation}\label{chap:install}

\section{Requirements}

\subsection{Hardware}
The file server makes use of Parsytec's MSC-board which offers at least 4 MBytes
RAM and a T805 Transputer for processing. The PFS will run on this processor and
builds up its buffer cache structures by making use of the MSC's memory. A
suitable SCSI-device with a capacity of up to 2 GBytes can be added. If you
have used HFS v1.x of PFS v2.0.x up to now, see section \ref{sec:hwupdate} for
necessary hardware updates.

\begin{note}
  If you have used an old version of the MSC board up to now, please refer 
  section \ref{sec:hwupdate} for an update of your hardware.
\end{note}

Figure \ref{pic:network} demonstrates how the PFS is integrated into a system which
makes use of the MSC and a SCSI-device.
\begin{figure}[htb]
  \small
  \begin{figure}[htb]
  \small
  \begin{figure}[htb]
  \small
  \input{network.pic}
  \caption{Sample network}
  \label{pic:network}
\end{figure}

  \caption{Sample network}
  \label{pic:network}
\end{figure}

  \caption{Sample network}
  \label{pic:network}
\end{figure}


\subsection{Software}

To work with the PFS properly, the ``Helios\index{Helios} distributed
operating system'' is required. The software is able to run with Helios
version 1.2 and upper.

\section{Copying files}

The Parsytec File System is shipped on one 5,25'' (HD) disc in IBM-PC 
compatible format respectively on a quarter inch streamer tape. Your disc/tape
is expected to contain the files listed in {\tt contents.pfs}. Copy all files
from the subdirectories of the distribution disc to the associated directories
of the host's file system. If you use a PC, this can be done for example by

\fbox{\tt >xcopy a: c:$\backslash$helios \slash s} \hfill (under MS-DOS)

or

\fbox{\tt \% cp -r \slash a\slash * \slash helios} \hfill (under Helios)

SUN users perform

{\tt
  \begin{tabular}{|l|}
    \hline
    cd /helios        \\
    tar xvf /dev/rst0 \\
    \hline
  \end{tabular}
}

\section{Device configuration} \label{sec:di}

This section describes how to establish your PFS. For that reason we will
create the file \HEDIS\ and compile it via the {\tt gdi}-command to \HEDI. 
The file server will use this file to get technical information about your 
hardware and about the logical structure you want to give to your file system.

\begin{note}

  \begin{itemize}
  
    \item In addition to \DI, \SI\ contains information about the connected
          SCSI devices. Refer chapter \ref{chap:msc} if you want to add SCSI
          devices of a not yet supported type.

    \item If you have used HFS v1.x or PFS v2.0.x before, see chapter 
          \ref{chap:upgrade} and execute the changes do your \DIS\ first.

  \end{itemize}

\end{note}

\subsection{Conditions}

There are three different kinds of information you have to write down in your
\HEDIS\ file (henceforth called \DI\ to save space):

\begin{itemize}

  \item technical facts concerning your hardware (e g a Winchester's 
        SCSI-address). For these you should be armed with a guide to your
        devices.

  \item strategical information about the method you want to manage your 
        file system (e g the size and apportionment of your buffer cache). For 
        these you should consult the file {\tt bible.src} (introduced below)
        to get some convenient standard values.

  \item arbitrary but necessary information about your taste to handle your 
        volumes. For this you should have (if you prefer to name your volumes 
        with your friends'\index{friend} names) at least five friends if you
        want to establish five volumes, for example.

\end{itemize}

\begin{note}
  The Parsytec File System, once configured, allows you to forget detailed
  information about the SCSI drives connected to the MSC board. You only work
  with some volumes (data pools with a name), whereat there can be some
  logical volumes on one hardware drive, or some partitions of different
  drives can be combined to one volume. Since \DI\ is the essential way to
  configure your file server, you should not do a `quick and dirty'-hack. As
  well as some values cannot be changed without loss of data, you will pay for
  every minute you save creating the \DI\ with much more time in your daily
  work, because inconsiderate partition of your buffer cache can slow down
  data transfer significantly, for example.
\end{note}

\subsection{General structure}

The \DI\ contains a sequence of entries consisting of a keyword, a name, and
a description. The keywords currently supported are {\it fileserver} and
{\it discdevice}. The name is used to identify the entry and has no other meaning.
The description is enclosed in braces (`\verb|{|' and
`\verb|}|'\index{$\lbrace \rbrace$}) and consists of a sequence of
keyword\slash value pairs. The character `{\tt \#}'\index{\#} introduces a
comment which extends to the end of the line. Keywords expect to be given a
value of a certain type, this may be either a name, a file name, a number or
a description. Names consist of a leading alphabetic character followed by
alphanumerics or periods\index{.}. File names are a sequence of names separated
by slashes (`{\tt /}'\index{/}). Numbers are sequences of numerals preceded by 
an optional minus sign\index{-}. By default, numbers are interpreted in decimal,
but the standard C-syntax may be used to provide values in octal\index{0} or 
hexadecimal\index{0@0x}.

\subsection{The developer's box - an example}

To make you familiar with creating your own \DI\ we will watch step by step
the formation of the \DI\ that was used by the programmer of the file server to
test his program.

\begin{note}
  It's easier to describe the configuration buttom-up, although the notation
  in our \DI\ is top-down.
\end{note}

\subsection{The hardware resources}

\subsubsection{Description of the equipment}

On our desktop we see a big box containing

\begin{tabular}{lp{9cm}}
  CDC WREN VI       & a 600 MB Winchester drive,                           \\
  Exabyte EXB-8200  & an 8mm Cartrige Tape Subsystem,                      \\
  Sony SMO-D501     & a rewritable 5.25'' optical disc drive, connected to
                      a SMO-C501 SCSI controller,                          \\
  Rodime 40         & a 40 MB Winchester drive,                            \\
  Tandberg TDC 3600 & a conventional tape streamer,                        \\
\end{tabular}

all SCSI devices.
First we want to tell our file server about these five hardware drives. We do
this by using the {\it drive}\index{drive@{\it drive}} keyword, followed by
information about the SCSI address ({\it id}\index{id@{\it id}}) and a
{\it type}\index{type@{\it type}} identification, both together enclosed in
braces. The value that is given to the {\it type} keyword refers to entries in
the \SI\ file, which is described more detailed in section \ref{sec:si}.

\begin{note}
  Similar to \DI, \SI\ can be changed by the user to make it possible to use
  non-standard hardware in nexus with the PFS, whereas `non-standard' means
  SCSI devices `not delivered by Parsytec'. All drives that you purchase at
  Parsytec are supported by the standard \SI\ (which is updated in case of
  enlargement of our hardware assortment). In the following all {\it type}
  values refer to the original \SIS\ and have to be corrected if that file is
  changed.
\end{note}

Let's start creating \DI\ with writing down the drive list (that consists of
five {\it drive} blocks which are numbered by the file server automatically):

\index{SCSI!address}
\index{logicalunitnumber@Logical Unit Number}
\index{LUN}

\begin{listing}
  \begin{verbatim}
# Remember, a #-character introduces a comment which ends here ->
# The first drive is the Rodime 40MB. It will be referenced by number 0 (in
# our drive list) by the file server. (NOTE: the sequence of the drives
# stipulates their numbering)

drive       # the drive keyword indicates the start of some hardware device 
            # information
{           # the opening brace
  id   0x00 # this drive occupies SCSI address 0 (represented by the first
            # hexadecimal digit), and Logical Unit Number 0 (specified by
            # by the latter hexdigit)
  type 1    # this drive is of type 1 (which means Rodime drive when working
            # with standard scsiinfo)
}           # the closing brace

# The second drive is the WREN VI. Since its description follows drive 0's
# specification, it will get number 1 in the drive list

drive       # hey file server, here's an info about my hardware
{           # from here
  id   0x10 #   SCSI address 1, LUN 0
  type 0    #   getting information from device block #0 in scsiinfo, which
            #   represents a WREN VI device
}           # to here

# Optical disc drive, number 2 in drive list

drive
{
  id   0x30 # SCSI address 3, LUN 0
  type 2    # optical disc
}

# Number 3:

drive{id 0x40 type 3}#this version to declare an Exabyte streamer works too,
#of course, but doesn't satisfy our esthetic necessaries

# Number 4:

drive         # so let's return to this pretty styled version to explain the
{             # fifth drive, the Tandberg tape streamer
  id    0x50  # at SCSI address 5, LUN 0
  type  4     # described in scsiinfo at position #4
}
  \end{verbatim}
\end{listing}

\begin{note}
  If you remove one of your drives and delete its description, all following
  {\it drive} blocks will get a new number automatically. This has
  consequences in the {\it partition} block (see below).
\end{note}

Figure \ref{pic:dev2vol1} reflects the situation after writing down all drive
descriptions.\index{SCSI!bus}
\begin{figure}[htb]
  \footnotesize \sf
  \begin{figure}[htb]
  \footnotesize \sf
  \begin{figure}[htb]
  \footnotesize \sf
  \input{dev2vol1.pic}
  \caption{Device connection to the SCSI bus}
  \label{pic:dev2vol1}
\end{figure}

  \caption{Device connection to the SCSI bus}
  \label{pic:dev2vol1}
\end{figure}

  \caption{Device connection to the SCSI bus}
  \label{pic:dev2vol1}
\end{figure}


\subsubsection{From drives to partitions}

Now, the file server knows the SCSI address, Logical Unit Number, and the
drive {\it type} of every SCSI device in our system. The second thing we want
to do is to install a low level logical structure on all the drives. This
structure is a division in subsections, called partitions. A
{\it partition}\index{partition@{\it partition}} (use this keyword to unite
the following information) is defined by the drive number where it takes
place, the first and the last block (the corresponding keywords are
{\it drive}\index{drive@{\it drive}}, {\it start}\index{start@{\it start}} and
{\it end}\index{end@{\it end}}).
Afterwards, partitions can be combined to larger structure, called volumes.
These volumes are the data pools the user works with.

\begin{note}

  \begin{itemize}

    \item The {\it drive} keyword must be given a value.

    \item The {\it start} and {\it end} keywords are only allowed if the medium that keeps
          the partition is structured (can be random accessed) and is
          unremoveable (the medium cannot be changed). In other terms: you
          need a Winchester drive to use {\it start} and {\it end} (which is given the
          last block by default).

    \item The values of {\it start} and {\it end} are interpreted in {\it addressing} size. This
          keyword is described below.

    \item As the {\it drive} list, the {\it partition} list will be numbered by the
          file server automatically - with the same consequences when deleting
          one {\it partition} block (following partitons' number will decrease by
          one, so that an eror occurs in a {\it volume} definition block, see
          below).

  \end{itemize}

\end{note}

The partitions:

\begin{listing}
  \begin{verbatim}
# The first partition occupies the first 20,000 blocks of drive 0 (that 
# was the Rodime 40MB).

partition      # number 0 in partition list
{
  drive 0      # this partition takes place on drive 0,
  start 0      # starts at block 0 and
  end   19999  # ends at block 19999.
               # NOTE:
               #   The size of a block is defined by yourself. You can do
               #   that by giving a value to the keyword addressing as 
               #   described below. So, if addressing equals 1024, this 
               #   partition (number 0) takes the first 20MB on drive 0.
}

partition      # number 1 in partition list
{
  drive 0      # the same drive as partition 0
  start 20000  # starts beyond partition 0 (that ended at block 19,999)
               # Since we want to get the rest of drive 0, the end keyword is
               # obsolete.
}

# Now we start to devide drive 1 (the WREN VI) in different partitions:
# the first with 20,000 blocks (0 - 19,999),
 
partition      # number 2 in list
{
  drive 1      
  start 0
  end   19999
}

# the two following with each 100,000 blocks

partition      # number 3
{
  drive 1
  start 20000
  end   119999
} 

partition      # number 4
{
  drive 1
  start 120000
  end   219999
} 

# and the 6th that takes the rest of drive 1.

partition      # number 5
{
  drive 1
  start 220000
}
 
# The Tandberg streamer occupies a complete partition

partition     # number 6
{
  drive 4     # a streamer is changable and not structured, so we have to give
              # up to divide it in smaller partitions
}
        
# as well as the optical disc

partition     # number 7
{
  drive 2     # an optical disc is changable, so the start and end keywords
              # are not allowed
}

# and the Exabyte streamer.

partition     # number 8
{
  drive 3
}
  \end{verbatim}
\end{listing}

\subsubsection{A complete discdevice} \label{sec:complete_dd}

Enclose the nine {\it partition} blocks and the four {\it drive} blocks
together in braces and place the keyword
{\it discdevice}\index{discdevice@{\it discdevice}}, followed by a name for
the group of drive resources, in front of that block. There are three
specifications left to add in the {\it discdevice} block to complete the hardware
declaration: the {\it name}\index{name@{\it name}} of the file containing the
device drivers (placed in the {\tt /helios/lib} directory), the address of the
SCSI {\it controller}\index{controller@{\it controller}} and the addressing
size for the devices in this group (following the keyword
{\it addressing}\index{addressing@{\it addressing}}, interpreted in byte).
Let's have a look at our complete {\it discdevice} block:

\begin{listing}
  \begin{verbatim}
discdevice msc21        # Mass Storage Controller, driver version 2.1
{
  name        msc21.dev # FIXED and expanded to /helios/lib/msc21.dev
  controller  0x70      # the controller's SCSI address and LUN
  addressing  1024      # FIXED: we want all disc devices to handle with an
                        # addressing size of one KByte
  partition             # number 0
  {
    :
  }

    :                    see above
    :

  drive                # number 0     
  {
    :
  }
    
    :                    see above
    :
}                      # end of discdevice msc21 block
  \end{verbatim}
\end{listing}

See figure \ref{pic:dev2vol2} for the effects.
\begin{figure}[htb]
  \footnotesize \sf
  \begin{figure}[htb]
  \footnotesize \sf
  \begin{figure}[htb]
  \footnotesize \sf
  \input{dev2vol2.pic}
  \caption{Partitioning of the drives}
  \label{pic:dev2vol2}
\end{figure}

  \caption{Partitioning of the drives}
  \label{pic:dev2vol2}
\end{figure}

  \caption{Partitioning of the drives}
  \label{pic:dev2vol2}
\end{figure}


\subsection{Piling up the file server}

\subsubsection{From partitions to volumes}

At this point we want to attach our nine partitions to seven volumes, whereby
two volumes consist of each two partitions and the other volumes claim only 
one. Since the names of my friends are not to you, the seven volume names will
be provided by the bible. In addition to the {\it name}\index{name@{\it name}}
of a {\it volume}\index{volume@{\it volume}} you must enter at least two
keywords and their values: {\it partition}\index{partition@{\it partition}}
and {\it type}\index{type@{\it type}}. The simpliest volume blocks look like
these:

\index{pride@{\tt /pride}}
\index{raw@{\it raw}}
\index{structured@{\it structured}}
\begin{listing}
  \begin{verbatim}
volume            # number 0
{
  name      pride
  partition 8     # the complete Exabyte streamer
  type      raw   # in opposite to partitions on random accessable drives like
                  # optical disc drives or Winchesters which are of type 
                  # structured
}

  \end{verbatim}
\end{listing}
\index{anger@{\tt /anger}}
\begin{listing}
  \begin{verbatim}
volume            # number 1
{
  name      anger 
  partition 6     # This is the Tandberg device
  type      raw
}
  \end{verbatim}
\end{listing}

The next volume makes use of the {\it minfree}\index{minfree@{\it minfree}}
keyword. This causes your file server to leave some blocks free for the
{\tt /lost+found}\index{lostandfound@{\tt /lost+found}} directory that will be
created by the checker.

\index{covetousness@{\tt /covetousness}}
\begin{listing}
  \begin{verbatim}
volume                    # number 2
{
  name      covetousness
  partition 7             # optical disc drive (complete)
  minfree   100           # all but 100 blocks are used by the file server
                          # NOTE:
                          #   The value of minfree is interpreted in the
                          #   fileserver's blocksize (see below) and NOT in
                          #   discdevice's addressing size. 
  type      structured
}
  \end{verbatim}
\end{listing}

Let's try to construct our first Winchester volume now using two more optional
keywords, {\it cgsize}\index{cgsize@{\it cgsize}} and
{\it ncg}\index{ncg@{\it ncg}}. The blocks of a partition are combined to
larger structures called cylinder groups. You are able to choose the number of
blocks per cylinder group (via {\it cgsize}) and/or the number of cylinder
groups per {\it partition} (via {\it ncg}). The remaining value(s) is\slash
are calculated by the file server.

\begin{note}

  \begin{itemize}

    \item There are maxima for both, the {\it cgsize} and the {\it ncg} values
          (see below).

    \item It's no error if the product of {\it cgsize} and {\it ncg} is
          smaller than the number of blocks your partitions offer. The
          remaining blocks are {\bf not} used.

  \end{itemize}

\end{note}

\index{lust@{\tt /lust}}
\begin{listing}
  \begin{verbatim}
volume                 # number 3
{
  name      lust       
  partition 4          # in the middle of WREN
  cgsize    256        # One cylinder group consists of 256 blocks.
                       # NOTE:
                       #   -Once more: the size of THIS blocks are fixed by
                       #    the blocksize keyword.
                       #   -The ncg keyword was not used, its value will be
                       #    calculated.
                       #   -The maximum for cgsize is 3072.
  minfree   100
  type      structured 
}
  \end{verbatim}
\end{listing}

Ok, well done. Immediately the next one (using both cylinder group parameters):

\index{gluttony@{\tt /gluttony}}
\begin{listing}
  \begin{verbatim}
volume                 # number 4
{
  name      gluttony
  partition 1          # second part of the Rodime
  cgsize    256        # assuming that the blocksize is 4096 bytes, a
                       # cylinder group occupies 1024 KBytes,
  ncg       20         # so volume 4 claims 20 MBytes
                       # NOTE:
                       #   -Partition 1 must offer that 20 MBytes, of course,
                       #    but here it does
                       #   -The maximum for ncg is 300.
  minfree   100
  type      structured 
}
  \end{verbatim}
\end{listing}

The two-partitions-on-one-volume-version:

\index{envy@{\tt /envy}}
\begin{listing}
  \begin{verbatim}
volume                 # number 5
{
  name      envy
  partition 3          # second part of the WREN
  partition 5          # fourth part of the WREN
                       # neither the ncg nor the cgsize keyword is used, the
                       # file server has to do the job
  minfree   100
  type      structured 
}
  \end{verbatim}
\end{listing}

Last but not least: two partitions on two drives combined to one volume:

\index{sloth@{\tt /sloth}}
\begin{listing}
  \begin{verbatim}
volume                 # number 6
{
  name      sloth
  partition 0          # Rodime's first
  partition 2          # WREN's first
  ncg       40         # We want the file server to install 40 cylinder
                       # groups.
                       # NOTE:
                       #   -The cgsize keyword was not used, its value will
                       #    be calculated.
  minfree   100
  type      structured 
}
  \end{verbatim}
\end{listing}

The volume structure is shown in figure \ref{pic:dev2vol3}---I hope we
survive.
\begin{figure}[htb]
  \footnotesize \sf
  \begin{figure}[htb]
  \footnotesize \sf
  \begin{figure}[htb]
  \footnotesize \sf
  \input{dev2vol3.pic}
  \caption{Combining partitions to volumes}
  \label{pic:dev2vol3}
\end{figure}

  \caption{Combining partitions to volumes}
  \label{pic:dev2vol3}
\end{figure}

  \caption{Combining partitions to volumes}
  \label{pic:dev2vol3}
\end{figure}



\subsubsection{Complete?}

Almost. Two kind of information are left to write down: the type of
{\it discdevice} you want to handle with your file server (following the
keyword {\it device}\index{device@{\it device}}) and some information about
your buffer cache's apportionment. There are three different kinds of
packages, and it's your's to determine their number. Here's the complete
{\it fileserver}\index{fileserver@{\it fileserver}} block:

\index{blocksize@{\it blocksize}}
\index{syncop@{\it syncop}}
\index{smallpkt@{\it smallpkt}}
\index{mediumpkt@{\it mediumpkt}}
\index{hugepkt@{\it hugepkt}}
\index{smallcount@{\it smallcount}}
\index{mediumcount@{\it mediumcount}}
\index{hugecount@{\it hugecount}}

\begin{listing}
  \begin{verbatim}
fileserver msc21
{
  device      msc21 # use the discdevice msc21 block (defined above)
  blocksize   4096  # FIXED to 4 KBytes
  syncop      0     # 0 turns the synchronous mode OFF (1 turns ON).
                    # In Synchrous mode all data are written on disc
                    # immediately, else the sync process does that job every
                    # 20 seconds.
                    # NOTE:
                    #   The syncop switch concerns user data. System data
                    #   (like directory information) are always written at
                    #   once. 
  smallpkt    1     # FIXED to 1  -- blocks per small cache packet
  mediumpkt   4     # FIXED to 4  -- blocks per medium cache packet
  hugepkt     16    # FIXED to 16 -- blocks per huge cache packet
  smallcount  20    # number of small  cache packets (1 <= smallcount  <= 30)
  mediumcount 8     # number of medium cache packets (1 <= mediumcount <= 15)
  hugecount   28    # number of huge   cache packets (1 <= hugecount   <= 42)
                    # NOTE:
                    #   The number of blocks that are reserved for the
                    #   buffer cache must be <= 750.
  volume            # number 0
  {
    :
  }
    :                 see above
    :
    
}
  \end{verbatim}
\end{listing}

\section{Running the multi-volume file server}\label{sec:runpfs}

The following example checklist should help you to setup a new file system and 
work with it. For this example we use a network with a node named ``MSC''
which offers a T805 and at least 4 MBytes RAM to keep the file server with all
data structures. We assume that you've already copied the files from the
distribution disc as described at the beginning of this chapter.

\begin{note}
  There are two ways to start the Parsytec File System. Note that the the MSC
  node must be of attribute HELIOS (declared in your resource map) on both
  cases.

  \begin{enumerate}

    \item To get the first experience, it's recommended to start it via
          {\tt remote} from your shell. This allows the execution of the file
          server under the full control of the user.


    \item Later, if everything is working well, the file server is capable of
          getting started in the general Helios startup sequence. The shell
          script \HE\slash {\tt startrc} is looking for {\tt pfsbook} and
          {\tt pfsrc} (both placed in the \HE\ directory) and excutes them.
          {\tt pfsbook} might look like

          {\tt
  \begin{tabular}{|l|}
    \hline \tt
    domain get /MSC  \\
    domain book /MSC \\
    \hline
  \end{tabular}
}


          whereas {\tt pfsrc} could be

          {\tt
  \begin{tabular}{|l|}
    \hline \tt
    test -d MSC                                   \\
    if (0 == \$status)                            \\
    \ \ \ \ remote -d MSC fs msc21                \\
    else                                          \\ 
    \ \ \ \ echo "No processor MSC to start PFS." \\
    endif                                         \\
    \hline
  \end{tabular}
}


          Creating this two files will guarantee that the file server is up
          and running when the user login prompt appears.
  \end{enumerate}

\end{note}

\subsection{Startup}

Boot the Helios network as usual

\fbox{\tt > server} \hfill (on PC systems) \\

or

\fbox{\tt \% helios} \hfill (on SUN systems)

After logging in, you can prepare an empty file system based on the
information which is kept in \DI. To change some of the file system
parameters, you have to edit \DIS\ as described above and recompile it with
the {\tt gdi}\index{gdi@{\tt gdi}}-command.

{\tt
  \begin{tabular}{|l|}
    \hline
    \% pushd /helios/etc       \\
    \% emacs devinfo.src       \\
    \                          \\
    \ \ \ \ \ <editing>        \\
    \                          \\
    \% gdi devinfo.src devinfo \\
    \% popd                    \\
    \hline
  \end{tabular}
}

If the \DI\ is successfully compiled, we start the file server on our
MSC-board via

\fbox{\tt \% remote -d MSC fs -f msc21}

Now the server is active and all volumes specified in the \DI\ are waiting for
being loaded. When using a volume the first time, an empty file system must be
created. We edit {\tt ldvol1st}\index{ldvol1st@{\tt ldvol1st}}

\fbox{\tt \% emacs ldvol1st}

{\tt
  \begin{tabular}{|l|}
    \hline \tt
    load -m \$1 \\
    makefs \$1  \\
    \hline
  \end{tabular}
}


and start it for every structured volume:

{\tt
  \begin{tabular}{|l|}
    \hline \tt
    \% ldvol1st /covetousness  \\
    \% ldvol1st /lust          \\
    \% ldvol1st /gluttony      \\
    \% ldvol1st /envy          \\
    \% ldvol1st /sloth         \\
    \hline
  \end{tabular}
}

\begin{note}
   Do not run {\tt ldvol1st} if your volume contains any data. To ease daily
   work we create the {\tt loadall}\index{loadall@{\tt loadall}} shell script

   \fbox{\tt \% emacs loadall}

   {\tt 
  \begin{tabular}{|l|}
    \hline \tt
    load /pride        \\
    load /covetousness \\
    load /lust         \\
    load /anger        \\
    load /gluttony     \\
    load /envy         \\
    load /pride        \\
    \hline
  \end{tabular}
}

   and start its execution in every future session.
\end{note}

Now, our network could look like this:

\begin{tabbing}
/Cluster/MSC\=/tasks\= \kill
/Cluster/MSC/pride/\ldots        \\
         \> /covetousness/\ldots \\
         \> /lust/\ldots         \\
         \> /anger/\ldots        \\
         \> /gluttony/\ldots     \\
         \> /envy/\ldots         \\
         \> /sloth/\ldots        \\
         \> /tasks/loader        \\
         \>    \> /procman       \\
         \>    \> /fs            \\
         \> \vdots               \\
\end{tabbing}

All volumes are started, you are now able to run our copy-demo (see chapter
\ref{chap:benchmarks}), format\index{format@{\tt format}} an optical disc or
do what you are payed for. If you like to format a disc, create a script like

\fbox{\tt \% emacs formtopt}\index{formtopt@{\tt formtopt}}

{\tt
  \begin{tabular}{|l|}
    \hline
    unload \$1   \\
    load -m \$1 \\
    format \$1  \\
    makefs \$1  \\
    \hline
  \end{tabular}
}


\begin{note}
  Formatting an optical disc takes approximately 45 minutes. So be aware
  before typing

  \fbox{\tt \% formtopt /covetousness}
\end{note}

The last shell script we offer to you leads to terminating all your volumes
and thereby the complete file server.

\fbox{\tt \% emacs termall}\index{termall@{\tt termall}}

{\tt
  \begin{tabular}{|l|}
    \hline
    termvol /pride        \\
    termvol /covetousness \\
    termvol /lust         \\
    termvol /anger        \\
    termvol /gluttony     \\
    termvol /envy         \\
    termvol /pride        \\
    \hline
  \end{tabular}
}


If you like to sin again, restart the file server.
