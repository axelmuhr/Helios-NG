\chapter{Backups}

\section{Creating archives}

The Parsytec tape archive program, {\tt ptar}\index{ptar@{\tt ptar}}, is a tool 
to store copies of a file or a group of files in an archive. This archive may be 
written directly to tape, stored as a file or sent through a pipe to another 
program, e g {\tt compress}. {\tt ptar} can also be used to list the files in an 
archive or to extract the files in the archive.

\subsection{What is an archive?}\index{archive}

An archive describes the names and contents of the constituent files. 
Archives are basically files, but may be written to and read from a tape. The 
format used by {\tt ptar} is compatible with the standard tar format, and 
archives may be sent to other machines even though they run different 
operating systems. Piping one {\tt ptar} to another is an easy way to copy a 
directory's contents from one place on a disc to another, hereby preserving 
the dates, modes and link-structure of all the files within.

\section{Argument syntax}

The full syntax of the {\tt ptar} command is as follows:

\fbox{\tt \% ptar <Options> [<Files>]}

Options and file names may be freely mixed, because each argument starting
with `{\tt -}' is considered to be an option argument. A single option
argument may contain several options, of which the last one may expect a 
parameter. This parameter should come immediately after the option, possibly 
separated from it by a space. It is not feasable to put more than one 
parameterised option into a single argument, since the rest of the argument 
following the first parameterised option is regarded as its parameter value.

\section{Operation modes}

{\tt ptar} is used to create an archive, to extract files from an archive or 
to list the contents of an archive as well. Each time you run {\tt ptar}, you 
must specify exactly one of these operation modes which should be the first
option for {\tt ptar}. Other arguments are file names to work on, files to put
into the archive or the files to extract from it. If you don't specify any 
file name, the default will depend on the operation mode: when creating an
archive, all files in the current directory are used. When reading, listing
or comparing an archive, {\tt ptar} will work on all files in the archive. If 
a file name argument actually names a directory, then that directory, its 
files and all its subdirectories are used.

Here's a list of the {\tt ptar} operation modes:

\begin{itemize}

  \item {\tt -c} \\
        Create a new archive that contains all the files specified on the 
        command line.

  \item {\tt -d} \\
        Compare the files in the archive with their counterparts in the file 
        system. {\tt ptar} will report all differences in size, mode, access 
        matrix and contents. If a file exists in the archive but cannot be 
        found in the file system, it will report this. If you specify file 
        names, those files are compared with the archive and they all must 
        exist in the archive.

  \item {\tt -t} \\
        Display a list of the files in the archive. If you specify file names, 
        only those files will be mentioned if they exist in the archive.

  \item {\tt -x} \\
        Extract the specified files from the archive. If no file names are 
        given, all files from the archive will be extracted.

\end{itemize}

\section{Other options}

All other options are not compulsory. Some of them make sense with all modes,
while others should only be used with particular modes.

\subsection{General options}
These options are always meaningful, regardless of the operation mode.

\begin{itemize}

  \item {\tt -b <Number>} \\
        Use a blocking factor of {\tt <Number>} for the archive. The default 
        blocking factor is 20 blocks. When reading or writing the archive, 
        {\tt ptar} will always read from or write to the archive in blocks of 
        {\tt <Number>} * 512 bytes. Larger blocking factors result in better 
        throughput of data, but might reduce media utilisation.

  \item {\tt -C <Directory>} \\
        Change into the {\tt <Directory>} before continuing. This option is 
        usually interspersed with the files {\tt ptar} should work on. It is 
        especially useful when you have several widely spread files that you 
        want to store in the same direcctory. For example, 

        \fbox{\tt \%  ptar -c <FileHere1> <FileHere2> -C <OtherDir> <FileThere>}

        will place the files {\tt <FileHere1>} and {\tt <FileHere2>} from the 
        current directory in the archive followed by the file 
        {\tt <FileThere>} from the directory {\tt <OtherDir>}. Note that the 
        file {\tt <FileThere>} is recorded under the precise name 
        {\tt <FileThere>}, not as {\tt <OtherDir>/<FileThere>}. Thus, the 
        archive will contain three files which all appear to have come from 
        the same directory; if the archive is extracted with `{\tt ptar -x}',
        all three files will be created in the same directory. In contrast,

        \fbox{\tt \% ptar -c <FileHere1> <FileHere2> <OtherDir>/<FileThere>}

        will record the third file in the archive under the name 
        {\tt <OtherDir>/<FileThere>} so that, if `{\tt ptar -x}' is used, the
        third file will be created in a subdirectory named {\tt <OtherDir>}.

  \item {\tt -f <Filename>} \\
        Use {\tt <Filename>} as the name of the archive file. If no `{\tt -f}'
        option is given but the environment variable {\tt TARFILE} exists, its 
        value is used, otherwise the {\tt ptar} writes to the default 
        `{\tt tar.out}'. If the filename is `{\tt -}', {\tt ptar} writes to
        the standard output (when creating) or reads from the standard input 
        (when listing or extracting). Thus, {\tt ptar} can be used as the 
        head or the tail of a command pipeline.

  \item {\tt -M} \\
        Work on a multi-volume archive - an archive that will not fit on a 
        single medium used to hold it. When this option is used, {\tt ptar} 
        will not abort when it reaches the end of a medium. Instead, it will 
        ask to prepare a new volume.

  \item {\tt -N <Date>} \\
        Work only on files whose creation or modification dates are newer than 
        {\tt <Date>}. The main purpose is for creating an archive; then only 
        new files are written. If extracting, only newer files are extracted. 

  \item {\tt -R} \\
        Print, along with each message, the record number within the archive 
        where the message occurred. This option is especially useful when 
        reading damaged archives, since it helps to pinpoint the damaged 
        sections.

  \item {\tt -T <Filename>} \\
        Read a list of file names from {\tt <Filename>} and add them to the 
        list of files to work on. If {\tt <Filename>} is given as `{\tt -}',
        the list is read from standard input. Several `{\tt -T}' options may
        be given in the command line. Note that using both `{\tt -T -}' and
        `{\tt -f -}' will not work unless you have specified the `{\tt -c}'
        mode before.

  \item {\tt -v} \\
        This option causes {\tt ptar} to be verbose about the actions it is 
        taking. Normally {\tt ptar} does its work silently; this option 
        displays the name of each file {\tt ptar} treats. When used with the
        `{\tt -t}' option, {\tt ptar} prints full information about each file
        like `{\tt ls -l}'.

  \item {\tt -w} \\
        Wait for user confirmation before taking the specified action. 
        {\tt ptar} prints a message for each operation it intends to take, and 
        waits for a line of input. If your input line begins with `{\tt y}',
        the action is performed, otherwise it is skipped. This option can only 
        be used together with the `{\tt -f -}' option when an archive is
        created.

\end{itemize}

\subsection{Create options}
These options are used to control which files {\tt ptar} puts into an archive, 
or to control the format the archive is written in.

\begin{itemize}

  \item {\tt -h} \\
        Follow symbolic links as if they were normal files or directories and 
        archive the linked-to object. Normally, {\tt ptar} simply records the 
        presence of a symbolic link. If the linked-to object is encountered 
        again, a complete second copy will be written to the archive.

  \item {\tt -V <Name>} \\
        Write a volume header at the beginning of the archive. If this option 
        is used together with `{\tt -M}', each volume of the archive will have
        a header of `{\tt <Name>} Volume {\tt <N>}', whereas {\tt <N>} is the
        volume number.

\end{itemize}

\subsection{Extract options}
These options are useful for extracting files from the archive. 

\begin{itemize}

  \item {\tt -k} \\
        Keep existing files within the file system, do not overwrite them from 
        the archive.

  \item {\tt -m} \\
        Do not extract the modification and access dates from the archive. The 
        modification and access times will be the time of extraction.

  \item {\tt -p} \\
        Set the access matrix of extracted files exactly as recorded in the 
        archive. If this option is not used, access matrices will be set to 
        the default matrices for directories or files.

\end{itemize}

\section{Creating backups}

{\tt ptar} can be used to perform full or incremental backups. Backups should 
only be done when no other users or programs are modifying files in the 
file system. If files are modified while {\tt ptar} is making a backup, they 
may not be stored properly in the archive, in which case you won't be able to 
restore them if necessary. You should use the `{\tt -V}' option for full
backups to name the archive, so you can tell what an archive is even without
a label. For incremental backups, you will need to use the `{\tt -N <Date>}'
option to tell {\tt ptar} to only store files that have been modified or added 
since date, where date should be the date and time of the last full or 
incremental backup. A standard scheme is to do a monthly full backup and a 
weekly incremental backup of everything that has changed since the last 
monthly. Also, perform  a daily incremental dump of everything that has 
changed since the last monthly or weekly dump. Unless you are in a hurry (and 
trust in the {\tt ptar} program and your tapes), you should compare the backup 
with the file system after creation using the `{\tt ptar -d}' command to ensure
that the backup has been written properly. This will also detect cases where 
files have been modified while or just after being archived.
