\chapter{MSC device driver}\label{chap:msc}

This chapter gives you a short introduction to the SCSI bus, the internal
details of the MSC device driver and explains how to add new device
descriptions in the \SI\ file.

\section{The SCSI bus}\index{SCSI!bus}
The SCSI (Small Computer System Interface) as defined in the ANSI X3.131-1986
standard is a widely used interface between host computers and peripheral
devices, such as disc drives, tape streamers and printers. It supports data
rates up to 5 megabytes per second and provides some device independence for
host computers.

The Interface protocol differs between two types of participants: initiators
(devices capable of initiating an operation) and targets (devices capable of
responding to a request to perform an operation). Up to 8 devices of
different type may be connected to a single bus system, as the architecture
of SCSI includes a priority-based distributed arbitration scheme.

The Parsytec MSC board uses the single-ended version of the SCSI bus, thereby
allowing maximum cable lengths of 6.0 meters. The onboard WD 33C93A SCSI
controller satisfies the ANSI X3T9.2 specifications, with the exception of
device disconnection\slash reconnection.

\section{Internal details}

This section gives a deeper view into the MSC device driver. It examines the
inner structure of the driver, describes the usage of SCSI commands and
explains how the Helios requests are performed.

\subsection{Device driver internals}
A Helios device is a piece of software which contains code to access a
specific hardware device. It is loaded dynamically by the program which uses
it, typically a server, which then uses the hardware through a defined
interface and provides a GSP interface for it.

This MSC device driver is written to support the Parsytec File System in
connection with the MSC Mass Storage Controller board. It is highly optimised
for throughput and flexibility on the SCSI interface.

\subsection{Process structure}\index{processstructure@process structure}
The {\tt OpenDevice()} call loads the driver and calls its Open function.
The process structure after successful completion is shown in figure
\ref{pic:process1}.
\begin{figure}[htb]
  \begin{figure}[htb]
  \begin{figure}[htb]
  \input{process1.pic}
  \caption{Process structure after {\tt OpenDevice()}}
  \label{pic:process1}
\end{figure}

  \caption{Process structure after {\tt OpenDevice()}}
  \label{pic:process1}
\end{figure}

  \caption{Process structure after {\tt OpenDevice()}}
  \label{pic:process1}
\end{figure}


The driver forks a SCSI bus control process, which is responsible for
interaction with the controller hardware. This process runs at high priority
to reduce reaction times. Communication to this process uses hardware
channels to allow the use of the occam-style Alternate construct.
Opening a SCSI device generates a new device control process, which knows
about commands and sequences and handles single Helios requests. The
{\tt DevOperate()} function puts all forthcoming requests from the file system
into request queues, from which the device control process extracts them as 
they are handled individually. For random access devices, the device control
process uses a special algorithm to optimize head movement and to reduce seek
delays. Figure \ref{pic:process2} shows the process structure after openening
two SCSI devices.
\begin{figure}[htb]
  \begin{figure}[htb]
  \begin{figure}[htb]
  \input{process2.pic}
  \caption{Process structure after opening two SCSI devices}
  \label{pic:process2}
\end{figure}

  \caption{Process structure after opening two SCSI devices}
  \label{pic:process2}
\end{figure}

  \caption{Process structure after opening two SCSI devices}
  \label{pic:process2}
\end{figure}
\index{processstructure@process structure}

\subsection{Supported Helios requests}

Due to the demands of the Parsytec File System, the driver supports 8
different device requests. Originally, these requests refer to partitions,
but they are mapped onto physical drives within the device driver. Valid
requests are:

\begin{itemize}

  \item {\tt FG\_Open}\index{open@{\tt FG\_Open}} \\
        Open a SCSI device for further use. This request is implemented by
        executing a sequence of SCSI commands called the initialisation
        sequence, which has to be defined in \SI. If none of the commands
        fails and all necessary drive parameters have been evaluated, the
        device control process is forked off. Opening an already opened device
        will only succeed if the device is a fixed disc device, otherwise
        the request is deemed erroneous.

  \item {\tt FG\_Close}\index{close@{\tt FG\_Close}} \\
        Release an opened device and terminate its device control process. If
        the device has been opened more than once, the device control process
        stays alive until all users have closed the device.

  \item {\tt FG\_Read}\index{read@{\tt FG\_Read}} \\
        Read some data from the SCSI device. This request is directly mapped
        onto the Read SCSI command.

  \item {\tt FG\_Write}\index{write@{\tt FG\_Write}} \\
        Write some data to the SCSI device. Like Read, this request maps onto
        the Write SCSI command.

  \item {\tt FG\_GetSize}\index{getsize@{\tt FG\_GetSize}} \\
        Report the block size and the number of blocks for a partition. Using
        the information from the initialisation sequence, this request can be
        fulfilled without interaction with the SCSI device.

  \item {\tt FG\_Format}\index{format@{\tt FG\_Format}} \\
        Format a medium and prepare it for further usage. This request is
        implemented in several phases: First, the format sequence is executed,
        then the driver issues the Format SCSI command. After completion, the
        whole medium is verified and defective blocks are reassigned. For tape
        drives, the Format request may be used to erase a tape.

  \item {\tt FG\_WriteMark}\index{writemark@{\tt FG\_WriteMark}} \\
        Write filemarks to a tape using the Write Filemarks SCSI command.

  \item {\tt FG\_Seek}\index{seek@{\tt FG\_Seek}} \\
        Position a tape at the specified position. This request performs the
        Rewind SCSI command if both the position and size values of the
        request are zero. Otherwise, a Space SCSI command will be issued using
        the lowest 2 bits of the position value as the command code. The size
        value (positive or negative) specifies the range of medium movement.

\end{itemize}

\subsection{Error handling}
If a SCSI command returns a CHECK CONDITION status, the device driver issues
a Request Sense command and evaluates an error code from the replied sense
data. The class of this error code is interpreted in the following way:

\begin{itemize}

  \item {\tt EC\_Recover}\index{recover@{\tt EC\_Recover}} \\
        The driver retries the command up to five times before reporting the
        error code. For Read or Write SCSI commands, {\tt EC\_Recover} codes
        are suppressed to allow retries upon corrected ECC errors.

  \item {\tt EC\_Warn}\index{warn@{\tt EC\_Warn}} \\
        If a Read or Write SCSI command on a disc device returns an
        {\tt EC\_Warn} code, the driver tries to detect the number of the
        block which caused the problem. If possible, the defective block will
        be reassigned. In case of success, the driver returns a `Recoverable,
        broken Medium' error code.

  \item {\tt EC\_Error}\index{error@{\tt EC\_Error}},
        {\tt EC\_Fatal}\index{fatal@{\tt EC\_Fatal}} \\
        Error codes belonging to these classes cause the driver to abort the
        request immediately.

\end{itemize}

\section{Adding new SCSI devices} \label{sec:si}

This section gives a guideline to add new devices to the \SIS\ file. It
explains the principles of driver operation and presents tools for device
inspection and verification of a modified \SI\ file. Imagine you have got
your brand new DICKENS\index{DICKENS} Pickwick\index{Pickwick}-1836 drive and
you plan to use it with the Parsytec File System.
\begin{enumerate}

  \item You have to find out how the drive acts on the SCSI bus and how the 
        different commands work.

  \item You should add a new entry to your \SIS\ file to reflect the new
        drive.

  \item You should check whether the new drive can work together properly with 
        the other drives on the SCSI bus.

\end{enumerate}

\subsection{Required documents}
For testing the new drive, you should have the Reference Manual or User's
Manual at hand (the manual which defines the SCSI command subset supported by
the drive).

\subsection{Testing single SCSI commands}\label{sec:testdrv}
To explore the behaviour of a new SCSI drive, you might want to issue single
SCSI commands and have a look at the drive's respose. This can be achieved
using the {\tt testdrv}\index{testdrv@{\tt testdrv}} utility, which lets you
define the low-level parameters and data values for a SCSI request and shows
the returned data.

\begin{note}
  In most cases it's not necessary to test each SCSI command that's defined
  for a new drive. Usually you first create the new device description and try
  to run the {\tt testinfo} utility. Only if that fails, go step by step
  through the requests using the {\tt testdrv} utility.
\end{note}

It's presumed that you have connected the new device to the
SCSI bus, have adjusted an unequivocal SCSI address\index{SCSI!address} via
jumpers\index{jumper} or dip switches\index{dipswitch@dip switch} and have
installed\slash removed the SCSI terminators\index{SCSI!terminator}.

\begin{caution}
  The {\tt testdrv} utility does {\bf not} preserve any data previously
  contained in the tested drives. Prove that you have a complete set of backups
  for all tested drives.
\end{caution}

Start {\tt testdrv} on your MSC board via

\medskip
\fbox{\tt \% remote MSC testdrv <DiscDeviceName>}
\medskip

The screen changes to

\begin{screendump}
  \begin{verbatim}
Current Request parameters:

 1)      Target ID   : 0
 2)      Target LUN  : 0
 3)      Sector size : 0
 4)      Read        : 0
 5)      Blockmove   : 0
 6)      Timeout     : 0 sec
 7)      CDB size    : 0
 8)      CDB         : 00 00 00 00 00 00 00 00 00 00 00 00
 9)      Size        : 0
         Rest        : 0
         Status      : 0x0
 A)      Data        : 00 00 00 00 ...

 1..A: edit field  R: send request Q: quit
  \end{verbatim}
\end{screendump}


and you have the choice wether to construct a request by editing the different
fields, to perform the request or to leave the program. Let's send the first
request to a new SCSI device.  Type `{\tt 1}' to be asked for the

{\tt New Target ID :}

Type in the SCSI address of your new device. Change the values of Target LUN,
Sector size, Read and Timeout, and your screen is expected to look like

\input{screen2}

To be sure to work with the correct drive and not to destroy any useful data,
it's recommended to perform the Inquiry SCSI command first. On this account we
consult the ``Guide to the famous Pickwick drive'', and on page 69 we find the
entry shown in figure \ref{pic:pwguide1}.
%{\sf  \center
%  \bigskip
%  \begin{tabular}{ll}
%    Command:                  & INQUIRY \\
%    Group:                    & 0       \\
%    OP-code:                  & 12H     \\
%    Command descriptor block: &         \\
%  \end{tabular}
%
%  \footnotesize \sf
%  %{\sf  \center
%  \bigskip
%  \begin{tabular}{ll}
%    Command:                  & INQUIRY \\
%    Group:                    & 0       \\
%    OP-code:                  & 12H     \\
%    Command descriptor block: &         \\
%  \end{tabular}
%
%  \footnotesize \sf
%  %{\sf  \center
%  \bigskip
%  \begin{tabular}{ll}
%    Command:                  & INQUIRY \\
%    Group:                    & 0       \\
%    OP-code:                  & 12H     \\
%    Command descriptor block: &         \\
%  \end{tabular}
%
%  \footnotesize \sf
%  \input{pwguide1.pic}
%  \bigskip
%}
\begin{figure}[h]
  \sf
  \begin{tabular}{ll}
    Command:                  & INQUIRY \\
    Group:                    & 0       \\
    OP-code:                  & 12H     \\
    Command descriptor block: &         \\
  \end{tabular}
  {
    \footnotesize \sf
    \input{pwguide1.pic}
  }
  \caption{``Guide to the famous Pickwick drive'', page 69}
  \label{pic:pwguide1}
\end{figure}

%  \bigskip
%}
\begin{figure}[h]
  \sf
  \begin{tabular}{ll}
    Command:                  & INQUIRY \\
    Group:                    & 0       \\
    OP-code:                  & 12H     \\
    Command descriptor block: &         \\
  \end{tabular}
  {
    \footnotesize \sf
    %{\sf  \center
%  \bigskip
%  \begin{tabular}{ll}
%    Command:                  & INQUIRY \\
%    Group:                    & 0       \\
%    OP-code:                  & 12H     \\
%    Command descriptor block: &         \\
%  \end{tabular}
%
%  \footnotesize \sf
%  \input{pwguide1.pic}
%  \bigskip
%}
\begin{figure}[h]
  \sf
  \begin{tabular}{ll}
    Command:                  & INQUIRY \\
    Group:                    & 0       \\
    OP-code:                  & 12H     \\
    Command descriptor block: &         \\
  \end{tabular}
  {
    \footnotesize \sf
    \input{pwguide1.pic}
  }
  \caption{``Guide to the famous Pickwick drive'', page 69}
  \label{pic:pwguide1}
\end{figure}

  }
  \caption{``Guide to the famous Pickwick drive'', page 69}
  \label{pic:pwguide1}
\end{figure}

%  \bigskip
%}
\begin{figure}[h]
  \sf
  \begin{tabular}{ll}
    Command:                  & INQUIRY \\
    Group:                    & 0       \\
    OP-code:                  & 12H     \\
    Command descriptor block: &         \\
  \end{tabular}
  {
    \footnotesize \sf
    %{\sf  \center
%  \bigskip
%  \begin{tabular}{ll}
%    Command:                  & INQUIRY \\
%    Group:                    & 0       \\
%    OP-code:                  & 12H     \\
%    Command descriptor block: &         \\
%  \end{tabular}
%
%  \footnotesize \sf
%  %{\sf  \center
%  \bigskip
%  \begin{tabular}{ll}
%    Command:                  & INQUIRY \\
%    Group:                    & 0       \\
%    OP-code:                  & 12H     \\
%    Command descriptor block: &         \\
%  \end{tabular}
%
%  \footnotesize \sf
%  \input{pwguide1.pic}
%  \bigskip
%}
\begin{figure}[h]
  \sf
  \begin{tabular}{ll}
    Command:                  & INQUIRY \\
    Group:                    & 0       \\
    OP-code:                  & 12H     \\
    Command descriptor block: &         \\
  \end{tabular}
  {
    \footnotesize \sf
    \input{pwguide1.pic}
  }
  \caption{``Guide to the famous Pickwick drive'', page 69}
  \label{pic:pwguide1}
\end{figure}

%  \bigskip
%}
\begin{figure}[h]
  \sf
  \begin{tabular}{ll}
    Command:                  & INQUIRY \\
    Group:                    & 0       \\
    OP-code:                  & 12H     \\
    Command descriptor block: &         \\
  \end{tabular}
  {
    \footnotesize \sf
    %{\sf  \center
%  \bigskip
%  \begin{tabular}{ll}
%    Command:                  & INQUIRY \\
%    Group:                    & 0       \\
%    OP-code:                  & 12H     \\
%    Command descriptor block: &         \\
%  \end{tabular}
%
%  \footnotesize \sf
%  \input{pwguide1.pic}
%  \bigskip
%}
\begin{figure}[h]
  \sf
  \begin{tabular}{ll}
    Command:                  & INQUIRY \\
    Group:                    & 0       \\
    OP-code:                  & 12H     \\
    Command descriptor block: &         \\
  \end{tabular}
  {
    \footnotesize \sf
    \input{pwguide1.pic}
  }
  \caption{``Guide to the famous Pickwick drive'', page 69}
  \label{pic:pwguide1}
\end{figure}

  }
  \caption{``Guide to the famous Pickwick drive'', page 69}
  \label{pic:pwguide1}
\end{figure}

  }
  \caption{``Guide to the famous Pickwick drive'', page 69}
  \label{pic:pwguide1}
\end{figure}
\index{commanddescriptorblock@Command Descriptor Block}
\index{cdb@{\it cdb}}

That's all the information we need to give the correct values to CDB size, CDB
and size:

\begin{itemize}

  \item CDB size \\
        INQUIRY is a group 0 command, thus it's command descriptor block size 
        is 6.

  \item CDB \\
        \begin{itemize}

          \item The first byte of the CDB contains the opeartion code, which 
                is 12H in our example.

          \item The three most significant bits of the second byte have to 
                contain the LUN (which is zero here).

          \item The remaining bits of byte \#01 are reserved and must be set 
                to zero, as well as the bytes \#02 and \#03.

          \item The allocation length has to be fixed in byte \#04. Since we 
                want to make the drive to transfer the complete response to 
                the INQUIRY command, its maximum length has to be entered here
                (the Pickwick drive answers with 51 (33H) bytes).

          \item The first two bits of the last byte are not checked by the 
                drive (marked as X's in the CDB), so we can leave their values
                at 0. The next four bits are reserved and therefore expected
                to be 0. The last two bits are set to zero, too, because we
                don't want to perform several linked requests but only a
                single one.

        \end{itemize}

  \item Size \\
        The expected data is of length 33H, so {\tt testdrv} must reserve this 
        51 bytes for the response.

\end{itemize}

\begin{note}
  There are some useful keys when editing the CDB resp. the Data field:

  \begin{tabular}{ll}
    $\uparrow$,    Ctrl-P & Move cursor to previous line                 \\
    $\downarrow$,  Ctrl-N & Move cursor to next line                     \\
    $\leftarrow$,  Ctrl-B & Move cursor to previous byte\slash character \\
    $\rightarrow$, Ctrl-F & Move cursor to next byte\slash character     \\
    PgUp                  & Move cursor to previous page                 \\
    PgDn                  & Move cursor to next page                     \\
    Tab                   & Change to byte/ASCII area                    \\
    R                     & Reset all bytes  (set to 0x00)               \\
    S                     & Set all bytes    (set to 0xFF)               \\
    P                     & Preset all bytes (set to 0x00 \ldots 0xFF)   \\
  \end{tabular}

\end{note}

If your screen equals to

\begin{screendump}
  \begin{verbatim}
Current Request parameters:

 1)      Target ID   : 1
 2)      Target LUN  : 0
 3)      Sector size : 1024
 4)      Read        : 1
 5)      Blockmove   : 0
 6)      Timeout     : 5 sec
 7)      CDB size    : 6
 8)      CDB         : 12 00 00 00 33 00 00 00 00 00 00 00
 9)      Size        : 51
         Rest        : 0
         Status      : 0x0
 A)      Data        : 00 00 00 00 ...

 1..A: edit field  R: send request Q: quit
  \end{verbatim}
\end{screendump}


perform the request by pressing `{\tt r}'.
If everything was done in the right way, the value of status is {\tt 0x0} and
the Data field contains {\tt 01 80 01 00 \ldots} . Press `{\tt a}' to have a
look at the complete data block.

\begin{screendump}
  \begin{verbatim}
New Data

xxxxxx | x0 x1 x2 x3 x4 x5 x6 x7 x8 x9 xA cB xC xD xE xF | ascii
-------+-------------------------------------------------+-----------------
000000 | 01 80 01 00 2E 00 00 00 44 49 43 4B 45 4E 53 20 | ........DICKENS
000010 | 50 69 63 6B 77 69 63 6b 2D 31 38 33 36 20 20 20 | Pickwick-1836
000020 | 2D 30 37 3A 30 36 43 52 45 41 54 45 44 31 31 31 | -07:06CREATED111
000030 | 00 00 17 00 00 00 00 00 00 00 00 00 00 00 00 00 | ................
000040 | 00 00 00 00 00 00 00 00 00 00 00 00 00 00 00 00 | ................
000050 | 00 00 00 00 00 00 00 00 00 00 00 00 00 00 00 00 | ................
000060 | 00 00 00 00 00 00 00 00 00 00 00 00 00 00 00 00 | ................
000070 | 00 00 00 00 00 00 00 00 00 00 00 00 00 00 00 00 | ................
000080 | 00 00 00 00 00 00 00 00 00 00 00 00 00 00 00 00 | ................
000090 | 00 00 00 00 00 00 00 00 00 00 00 00 00 00 00 00 | ................
0000a0 | 00 00 00 00 00 00 00 00 00 00 00 00 00 00 00 00 | ................
0000b0 | 00 00 00 00 00 00 00 00 00 00 00 00 00 00 00 00 | ................
0000c0 | 00 00 00 00 00 00 00 00 00 00 00 00 00 00 00 00 | ................
0000d0 | 00 00 00 00 00 00 00 00 00 00 00 00 00 00 00 00 | ................
0000e0 | 00 00 00 00 00 00 00 00 00 00 00 00 00 00 00 00 | ................
0000f0 | 00 00 00 00 00 00 00 00 00 00 00 00 00 00 00 00 | ................
  \end{verbatim}
\end{screendump}


If the status is {\tt 0x1} (which means CHECK CONDITION), you have to perform
the REQUEST SENSE command to get information about the error that has occured.
That command is described in our handbook, too (see figure
\ref{pic:pwguide2}).

%{\sf \center
%  \bigskip
%  \begin{tabular}{ll}
%    Command:                  & REQUEST SENSE \\
%    Group:                    & 0             \\
%    OP-code:                  & 03H           \\
%    Command descriptor block: &               \\
%  \end{tabular}
%
%  \footnotesize \sf
%  %{\sf \center
%  \bigskip
%  \begin{tabular}{ll}
%    Command:                  & REQUEST SENSE \\
%    Group:                    & 0             \\
%    OP-code:                  & 03H           \\
%    Command descriptor block: &               \\
%  \end{tabular}
%
%  \footnotesize \sf
%  %{\sf \center
%  \bigskip
%  \begin{tabular}{ll}
%    Command:                  & REQUEST SENSE \\
%    Group:                    & 0             \\
%    OP-code:                  & 03H           \\
%    Command descriptor block: &               \\
%  \end{tabular}
%
%  \footnotesize \sf
%  \input{pwguide2.pic}
%  \bigskip
%}
\begin{figure}[h]
  \sf
  \begin{tabular}{ll}
    Command:                  & REQUEST SENSE \\
    Group:                    & 0             \\    
    OP-code:                  & 03H           \\
    Command descriptor block: &               \\
  \end{tabular}
  {
    \footnotesize \sf
    \input{pwguide2.pic}
  }
  \caption{``Guide to the famous Pickwick drive'', page 42}
  \label{pic:pwguide2}
\end{figure}

%  \bigskip
%}
\begin{figure}[h]
  \sf
  \begin{tabular}{ll}
    Command:                  & REQUEST SENSE \\
    Group:                    & 0             \\    
    OP-code:                  & 03H           \\
    Command descriptor block: &               \\
  \end{tabular}
  {
    \footnotesize \sf
    %{\sf \center
%  \bigskip
%  \begin{tabular}{ll}
%    Command:                  & REQUEST SENSE \\
%    Group:                    & 0             \\
%    OP-code:                  & 03H           \\
%    Command descriptor block: &               \\
%  \end{tabular}
%
%  \footnotesize \sf
%  \input{pwguide2.pic}
%  \bigskip
%}
\begin{figure}[h]
  \sf
  \begin{tabular}{ll}
    Command:                  & REQUEST SENSE \\
    Group:                    & 0             \\    
    OP-code:                  & 03H           \\
    Command descriptor block: &               \\
  \end{tabular}
  {
    \footnotesize \sf
    \input{pwguide2.pic}
  }
  \caption{``Guide to the famous Pickwick drive'', page 42}
  \label{pic:pwguide2}
\end{figure}

  }
  \caption{``Guide to the famous Pickwick drive'', page 42}
  \label{pic:pwguide2}
\end{figure}

%  \bigskip
%}
\begin{figure}[h]
  \sf
  \begin{tabular}{ll}
    Command:                  & REQUEST SENSE \\
    Group:                    & 0             \\    
    OP-code:                  & 03H           \\
    Command descriptor block: &               \\
  \end{tabular}
  {
    \footnotesize \sf
    %{\sf \center
%  \bigskip
%  \begin{tabular}{ll}
%    Command:                  & REQUEST SENSE \\
%    Group:                    & 0             \\
%    OP-code:                  & 03H           \\
%    Command descriptor block: &               \\
%  \end{tabular}
%
%  \footnotesize \sf
%  %{\sf \center
%  \bigskip
%  \begin{tabular}{ll}
%    Command:                  & REQUEST SENSE \\
%    Group:                    & 0             \\
%    OP-code:                  & 03H           \\
%    Command descriptor block: &               \\
%  \end{tabular}
%
%  \footnotesize \sf
%  \input{pwguide2.pic}
%  \bigskip
%}
\begin{figure}[h]
  \sf
  \begin{tabular}{ll}
    Command:                  & REQUEST SENSE \\
    Group:                    & 0             \\    
    OP-code:                  & 03H           \\
    Command descriptor block: &               \\
  \end{tabular}
  {
    \footnotesize \sf
    \input{pwguide2.pic}
  }
  \caption{``Guide to the famous Pickwick drive'', page 42}
  \label{pic:pwguide2}
\end{figure}

%  \bigskip
%}
\begin{figure}[h]
  \sf
  \begin{tabular}{ll}
    Command:                  & REQUEST SENSE \\
    Group:                    & 0             \\    
    OP-code:                  & 03H           \\
    Command descriptor block: &               \\
  \end{tabular}
  {
    \footnotesize \sf
    %{\sf \center
%  \bigskip
%  \begin{tabular}{ll}
%    Command:                  & REQUEST SENSE \\
%    Group:                    & 0             \\
%    OP-code:                  & 03H           \\
%    Command descriptor block: &               \\
%  \end{tabular}
%
%  \footnotesize \sf
%  \input{pwguide2.pic}
%  \bigskip
%}
\begin{figure}[h]
  \sf
  \begin{tabular}{ll}
    Command:                  & REQUEST SENSE \\
    Group:                    & 0             \\    
    OP-code:                  & 03H           \\
    Command descriptor block: &               \\
  \end{tabular}
  {
    \footnotesize \sf
    \input{pwguide2.pic}
  }
  \caption{``Guide to the famous Pickwick drive'', page 42}
  \label{pic:pwguide2}
\end{figure}

  }
  \caption{``Guide to the famous Pickwick drive'', page 42}
  \label{pic:pwguide2}
\end{figure}

  }
  \caption{``Guide to the famous Pickwick drive'', page 42}
  \label{pic:pwguide2}
\end{figure}


The CDB is followed by a description of the Extended Sense Byte Definitions
(including the Sense Key Value Definitions), which has a max. length of 26
(1AH) bytes. So we transform our screen to

\input{screen4}

and perform the request. Interpreting the received data byte by byte will
help us to find the error.

\subsection{Creating the \SI\ configuration file}\index{scsiinfo@\SI}

The device driver needs some technical information about the connected SCSI
devices. This information is provided in the file \HESI. Run {\tt gsi} with
the shipped file \HESIS, and {\tt gsi} will print a list of the supported
drives. \SIS\ is ASCII text and can be edited to add new device descriptions.
It is recommended to copy the file {\tt pattern.src} to {\tt <NewDevice>.src},
to change that copy afterwards like described in itself and in this manual,
and to append the new device description afterwards to \SI.

\subsubsection{General structure}
The \SIS\ file contains a sequence of entries consisting a keyword, a name and
a description. The currently supported keyword is
{\it device}\index{device@{\it device}}. The name is used to identify the
entry within the file and has no further meaning. The description is enclosed
in braces (`\verb|{|' and `\verb|}|'\index{$\lbrace \rbrace$}) and
consists of a sequence of keyword\slash value pairs. The character
`{\tt \#}'\index{\#} introduces a comment which extends to the end of the
line.

Each keyword expects a value of a certain type. This may either be a name, a
number, a description or a list. Names are either character sequences
enclosed in double quotes\index{''} or consist of a leading alphabetic
character followed by alphanumerics or underscores\index{\_}. Numbers are
sequences of numerals according to the standard C syntax\index{0}\index{0@0x},
optionally preceded by a minus sign\index{-}. Lists are sequences of numbers
in the range from 0 to 255, enclosed in braces.

See appendix \ref{app:si_syntax} for a formal description of the \SI\ file.
The next paragraphs will give you a top-down introduction.

\subsubsection{Toplevel}

A \SI\ file is a list of device descriptions and consequently has the
following form:

\begin{listing}
  \begin{verbatim}
device "Dickens Pickwick Series 18"  # This is the 1st device description,
{                                    # so you have to refer to it with
  :                                  # "type 0x0n" from the devinfo file
  :                                  # The gsi command which compiles
  :                                  # scsiinfo.src to scsiinfo will
  :                                  # report the string after the device
  :                                  # keyword when translating this
  :                                  # device description block. The
  :                                  # string has no other meaning.
}

:                                    # more devices...
:

device "Dickens Tupman Device"       # This is the last device description
{                                    # block within scsiinfo.
  :
}
  \end{verbatim}
\end{listing}
\index{Tupman}

\subsubsection{Device level}
Let's have a look at a typcial device description block:

\index{device@{\it device}}
\index{type@{\it type}}
\index{random@{\it random}}
\index{structured@{\it structured}}
\index{sequential@{\it sequential}}
\index{raw@{\it raw}}
\index{ident@{\it ident}}
\index{command@{\it command}}

\begin{listing}
  \begin{verbatim}
device "Dickens Pickwick Series 18"
{
  type        random # The device type specifies the set of mandatory
                     # commands for device operation. Other possible
                     # values are "structured" (wich means the same like
                     # "random" and points out a disc device), "raw" and
                     # "sequential" (wich both hint a tape device).

  ident       "DICKENS Pickwick-1836" # The specified name is expected as
                     # the reply to the Inquiry SCSI command and is used
                     # to verify the correct application of a description
                     # onto a device.

  command            # A SCSI command description as described below. A
  {                  # device-type dependent set of commands has to be
    :                # defined.
  }

  :

  command            # End of the command description list.
  {
    :
  }

  error              # An error condition which can be extracted from the
  {                  # reply to the Request Sense SCSI command. Conditions
    :                # which are not defined result in an unspecified
    :                # error.
  }

  :

  error              # End of the error description list.
  {
    :
  }

  request            # Both the Helios FG_Open and FG_Close requests are
  {                  # processed with a SCSI command sequence (which may
    :                # be empty). For the Helios FG_Format request, it is
    :                # also possible to define an initialisation sequence
    :                # to prepare the device for the Format command.
  }

  :

  request            # End of the request description list.
  {
   :
  }

}                    # End of the device description block.
  \end{verbatim}
\end{listing}

\index{error@{\it error}}
\index{request@{\it request}}

\subsubsection{Command/error/request level}

\paragraph{Commands}
A command description block is of the following form:

\index{command@{\it command}}
\index{name@{\it name}}
\index{read@{\it read}}
\index{no@{\it no}}
\index{blockmove@{\it blockmove}}
\index{yes@{\it yes}}
\index{cdbsize@{\it cdbsize}}

\begin{listing}
  \begin{verbatim}
command
{
  name       Mode_Select # The command name is used to identify the
                         # standard SCSI commands, which are mandatory for
                         # a device. Command names are also referred in
                         # sequence definitions. Command names are not
                         # case-sensitive, and underscores are treated as
                         # spaces.

  read       no          # The direction of data flow is necessary for
                         # correct driver operation.

  blockmove  yes         # The device driver can use a fast blockmove
                         # operation to speed up data transfer. This
                         # operation is only possible if the device
                         # performs sector transfers under all
                         # circumstances.

  cdbsize     6          # The size of the Command Descriptor Block in
                         # bytes. Default values are calculated for groups
                         # 0, 1, 2 and 6, other commands need this value.

  cdb                    # The Command Descriptor Block defines the SCSI
  {                      # command to be executed. For the Request Sense,
    :                    # Inquiry, Mode Sense, Read, Write, Verify, Write
    :                    # Filemark and Space SCSI commands position and
    :                    # size parameters are inserted at runtime. For
    :                    # all other commands, the CDB should contain the
    :                    # complete description including sizes.
  }

  datasize    0x10       # Only the Read, Write, Verify and Space SCSI
                         # commands use calculated data sizes. For all
                         # other commands, the transfer size in bytes
                         # should be supplied.

  data                   # Some SCSI commands, e g Mode Select, need
  {                      # user-supplied parameters. Up to datasize bytes
    :                    # may be provided here.
  }

}
  \end{verbatim}
\end{listing}
\index{cdb@{\it cdb}}
\index{datasize@{\it datasize}}
\index{data@{\it data}}

\subparagraph{Mandatory SCSI commands}
The device driver supports disc (random access) and tape (sequential access)
drives. Due to their different characteristics, both types use different SCSI
commands and different parameters. The following list names those SCSI
commands which have to be defined for either drive type.

\begin{itemize}

   \item Common commands \\
         These commands have to be defined for both device types:

         \begin{description}

           \item[Test Unit Ready]\index{testunitready@Test Unit Ready} checks
                whether a logical unit is ready for media access. It takes no
                parameter and has no reply.

                 \begin{listing}
                   \begin{verbatim}
command                               
{                                     
  name  Test_Unit_Ready         
  read  yes                     
  cdb   { 0x00 0x00             
          0x00 0x00             
          0x00 <VendorUnique> } 
        }                       
}
                   \end{verbatim}
                 \end{listing}

           \item[Request Sense]\index{requestsense@Request Sense}
                requests that the target shall send sense data to the
                initiator, thus providing additional information in case of
                errors. Both extended and nonextended sense data format is
                supported. The expected {\tt <DataSize>} should be supplied in
                \SI. Sense data is checked for known error conditions as
                defined in \SI\ (see below).

                 \begin{listing}
                   \begin{verbatim}
command
{
  name      Request_Sense
  read      yes
  cdb       { 0x03       0x00
              0x00       0x00
              <DataSize> <VendorUnique> }
  datasize  <DataSize>
}
                   \end{verbatim}
                 \end{listing}

           \item[Inquiry]\index{Inquiry}
                requests that information regarding the target parameters and
                identity is sent to the initiator. The expected
                {\tt <DataSize>} should be supplied in \SI. Inquiry data is
                used to check the device identity against the ident value of
                the \SI\ device entry. Further information regarding the
                device type and medium removability is also extracted.

                 \begin{listing}
                   \begin{verbatim}
command
{
  name      Inquiry
  read      yes
  cdb       { 0x12       0x00
              0x00       0x00
              <DataSize> <VendorUnique> }
  datasize  <DataSize>
}
                   \end{verbatim}
                 \end{listing}
         \end{description}

   \item Commands for direct-access device types \\
         Disc devices are always used in blocked mode. Therefore, a transfer 
         size value refers to a specific number of blocks. The following 
         commands are mandatory for disc devices:

         \begin{description}

           \item[Format Unit]\index{formatunit@Format Unit}
                ensures that the medium is formatted so that all data blocks
                are accessible. This does {\bf not} require a physical
                formatting, and there is no guarantee that the medium will be
                altered. The CDB is taken as is from \SI, any format mode and
                interleave parameters have to be named in the CDB definition
                there. Any defect lists should be supplied as data to the
                Helios request. The command has no reply.

                 \begin{listing}
                   \begin{verbatim}
command
{
  name        Format
  read        no
  cdb         { 0x04              <Mode>
                <VendorUnique>    <Interleave(MSB)>
                <Interleave(LSB)> <VendorUnique> }
}
                   \end{verbatim}
                 \end{listing}


           \item[Reassign Blocks]\index{reassignblocks@Reassign Blocks}
                requests the target to reassign defective blocks to unused
                spare blocks reserved for this purpose. The required defect
                list is generated by the device driver.

                 \begin{listing}
                   \begin{verbatim}

command
{
  name        Reassign_Blocks
  read        no
  cdb         { 0x07 0x00
                0x00 0x00
                0x00 <VendorUnique> }
}
                   \end{verbatim}
                 \end{listing}

           \item[Read]\index{Read!random} requests the target to transfer user
                data from the medium to the initiator. Both logical block 
                address and transfer size are calculated from the Helios request
                and inserted into the CDB.

                 \begin{listing}
                   \begin{verbatim}
command
{
 name        Read
 read        yes
 cdb         { 0x08 0x00
               0x00 0x00
               0x00 <VendorUnique> }
}
                   \end{verbatim}
                 \end{listing}

           \item[Write]\index{Write!random}
                requests the target to transfer user data from the initiator
                to the medium. Both logical block address and transfer size
                are calculated from the Helios request and inserted into the
                CDB.

                 \begin{listing}
                   \begin{verbatim}
command
{
  name        Write
  read        no
  cdb         { 0x0A 0x00
                0x00 0x00
                0x00 <VendorUnique> }
}
                   \end{verbatim}
                 \end{listing}


           \item[Mode Sense]\index{Mode Sense!random}
                provides a means for the target to report medium, unit or
                device parameters to the initiator. The expected
                {\tt <DataSize>} should be supplied in \SI. Mode data is used
                to check medium write protection. If a block descriptor is
                available, it will be used to get information about the number
                and the size of logical blocks on the medium.

                 \begin{listing}
                   \begin{verbatim}
command
{
  name      Mode_Sense
  read      yes
  cdb       { 0x1A       0x00
              0x00       0x00
              <DataSize> <VendorUnique> }
  datasize  <DataSize>
}
                   \end{verbatim}
                 \end{listing}

           \item[Read Capacity]\index{readcapacity@Read Capacity}
                causes the target to send information about the medium
                capacity to the initiator. Capacity data is used to determine
                the disc size, therefore the PMI bit of the CDB has to be
                cleared (set to zero).

                 \begin{listing}
                   \begin{verbatim}
command
{
  name      Read_Capacity
  read      yes
  cdb       { 0x25 0x00
              0x00 0x00
              0x00 0x00
              0x00 0x00
              0x00 <VendorUnique> }
  datasize  0x08
}
                   \end{verbatim}
                 \end{listing}

           \item[Verify]\index{Verify}
                command requests the target to verify data written on the
                medium. This shall be a medium verification (no data will be
                transferred for this command), and the BytChk bit of the CDB
                has to be cleared (set to zero).

                 \begin{listing}
                   \begin{verbatim}
command
{
  name        Verify
  read        no
  cdb         { 0x2F 0x00
                0x00 0x00
                0x00 0x00
                0x00 0x00
                0x00 <VendorUnique> }
}
                   \end{verbatim}
                 \end{listing}

         \end{description}

   \item Commands for sequential-access device types \\
         Tape devices may be used in blocked or nonblocked mode. The operating 
         mode has to be set up before the first medium access command is 
         issued, usually during the initialisation sequence. Transfer sizes 
         and position values in variable length mode will be in bytes. Now, 
         here are the commands which are needed for tape devices:

         \begin{description}

           \item[Rewind]\index{Rewind}
                requests that the target rewind the logical unit to the
                beginning-of-medium or load-point. Depending on the
                {\tt <Immediate>} bit, the status is returned as soon as the
                operation is initiated or after the operation is completed. It
                is recommended to clear this bit (set to zero).

                 \begin{listing}
                   \begin{verbatim}
command
{
  name        Rewind
  read        yes
  cdb         { 0x01 <Immediate>
                0x00 0x00
                0x00 <VendorUnique> }
}
                   \end{verbatim}
                 \end{listing}

           \item[Read]\index{Read!sequential} requests the target to transfer user data from the
                medium to the initiator. The transfer length is calculated
                from the Helios request and inserted into the CDB. The
                {\tt <Fixed>} bit has to be set according to the selected
                mode. In variable block mode, the transfer size should meet
                the block size found on the medium.

                 \begin{listing}
                   \begin{verbatim}
command
{
  name        Read
  read        yes
  cdb         { 0x08 <Fixed>
                0x00 0x00
                0x00 <VendorUnique> }
}
                   \end{verbatim}
                 \end{listing}

           \item[Write]\index{Write!sequential}
                command requests the target to transfer user data from the
                initiator to the medium. The transfer length is calculated
                from the Helios request and inserted into the CDB. The
                {\tt <Fixed>} bit has to be set according to the selected
                mode.

                 \begin{listing}
                   \begin{verbatim}
command
{
  name        Write
  read        no
  cdb         { 0x0A <Fixed>
                0x00 0x00
                0x00 <VendorUnique> }
}
                   \end{verbatim}
                 \end{listing}

           \item[Write Filemarks]\index{writefilemarks@Write Filemarks}
                causes a specified number of filemarks to be written to the
                medium.

                 \begin{listing}
                   \begin{verbatim}
command
{
  name        Write_Filemarks
  read        no
  cdb         { 0x10 0x00
                0x00 0x00
                0x00 <VendorUnique> }
}
                   \end{verbatim}
                 \end{listing}

           \item[Space]\index{Space}
                provides a variety of positioning functions determined by
                {\tt <Code>} and the count value, which is adapted to meet the
                block size if fixed mode is used.

                 \begin{listing}
                   \begin{verbatim}
command
{
  name        Space
  read        yes
  cdb         { 0x11 <Code>
                0x00 0x00
                0x00 <VendorUnique> }
}
                   \end{verbatim}
                 \end{listing}

           \item[Mode Sense]\index{Mode Sense!sequential}
                provides a means for the target to report medium, unit or
                device parameters to the initiator. The expected
                {\tt <DataSize>} should be supplied in \SI. Mode data is used
                to check medium write protection. If a block descriptor is
                available, it will be used to get information about the number
                and the size of logical blocks on the medium. A logical block
                size of zero marks variable length mode and is interpreted as
                a block length of one.

                 \begin{listing}
                   \begin{verbatim}
command
{
  name      Mode_Sense
  read      yes
  cdb       { 0x1A       0x00
              0x00       0x00
              <DataSize> <VendorUnique> }
  datasize  <DataSize>
}
                   \end{verbatim}
                 \end{listing}

         \end{description}

\end{itemize}

\subparagraph{Other SCSI commands}
Those commands described above are essential for driver operation, and the
presence of their definitions is checked by the {\tt gsi} compiler. Various
other commands may be defined in the \SI\ file, they can be used for device
initialisation or to prepare formatting. Some examples are the following
commands:

\begin{itemize}

  \item Commands for direct-access device types
        
        \begin{description}

          \item[Mode Select]\index{Mode Select!random}
               provides a means for the initiator to specify medium, unit or
               device parameters. Its parameter list has to be declared in
               \SI.

               \begin{listing}
                  \begin{verbatim}
command
{
  name      Mode_Select
  read      no
  cdb       { 0x15                  0x00
              0x00                  0x00
              <ParameterListLength> <VendorUnique> }
  datasize  <ParameterListLength>
  data      { 0x00
              <MediumType>
              0x00
              <BlockDescriptorLength>
              <BlockDescriptor(s)>
              <VendorUnique> }
}
                 \end{verbatim}
                \end{listing}

          \item[Start/Stop Unit]\index{startunit@Start Unit}
                                \index{stopunit@Stop Unit}
               causes the target to enable or disable the unit for further
               operation. A Winchester disc drive might spin down if a Stop
               Unit command is issued, hereby reducing its power consumption.
               The Start Unit command is typically part of the initialisation
               sequence to ensure correct drive operation.

               \begin{listing}
                  \begin{verbatim}
command
{
  name  Start_Unit
  read  no
  cdb   { 0x1B 0x00
          0x00 0x00
          0x01 <VendorUnique> }
}

command
{
  name  Stop_Unit
  read  no
  cdb   { 0x1B 0x00
          0x00 0x00
          0x00 <VendorUnique> }
}
                 \end{verbatim}
                \end{listing}

        \end{description}

  \item Commands for sequential-access device types

        \begin{description}

          \item[Mode Select]\index{Mode Select!sequential}
               provides a means for the initiator to specify medium, unit or
               device parameters. Its parameter list has to be declared in
               \SI.

               \begin{listing}
                  \begin{verbatim}
command
{
  name      Mode_Select
  read      no
  cdb       { 0x15                  0x00
              0x00                  0x00
              <ParameterListLength> <VendorUnique> }
  datasize  <ParameterListLength>
  data      { 0x00
              0x00
              <Mode>
              <BlockDescriptorLength>
              <BlockDescriptor(s)>
              <VendorUnique> }
}
                 \end{verbatim}
                \end{listing}

          \item[Erase]\index{Erase}
               requests part or all of the medium to be erased, which means
               that the medium shall appear as gap or unwritten. This command
               might be used as a replacement to the Format command to put a
               medium into a known state.

               \begin{listing}
                 \begin{verbatim}
command
{
  name  Format          # Do not use Erase!
  read  no
  cdb   { 0x19 <Long>
          0x00 0x00
          0x00 <VendorUnique> }
}
                 \end{verbatim}
                \end{listing}

          \item[Load/Unload]\index{Load}\index{Unload}
               causes the target to enable or disable the unit for further
               operation. A tape drive e g might eject the medium if a Unload
               command is issued, hereby reducing its power consumption. This
               SCSI command is mostly defined twice in \SI\ to load the medium
               within the initialisation sequence and to unload it after
               usage. It may also be used to request the re-tension function
               on peripheral devices that support this function.

               \begin{listing}
                 \begin{verbatim}
command
{
  name        Load
  read        no
  cdb         { 0x1B                 0x00
                0x00                 0x00
                (<ReTension> | 0x01) <VendorUnique> }
}

command
{
  name        Unload
  read        no
  cdb         { 0x1B        0x00
                0x00        0x00
                <ReTension> <VendorUnique> }
}
                  \end{verbatim}
                 \end{listing}

        \end{description}

  \item Commands for removable medium device types \\
        {\bf Prevent/Allow Medium Removal}
           \index{preventmediumremoval@Prevent Medium Removal}
           \index{allowmediumremoval@Allow Medium Removal}
        requests that the target should enable or disable the removal of the
        medium from the unit. Usually, this SCSI command is defined twice in
        \SI\ to distinguish between allowing and preventing medium removal.

        \begin{listing}
          \begin{verbatim}
command
{
  name        Prevent_Medium_Removal
  read        yes
  cdb         { 0x1E 0x00
                0x00 0x00
                0x01 <VendorUnique> }
}

command
{
  name        Allow_Medium_Removal
  read        yes
  cdb         { 0x1E 0x00
                0x00 0x00
                0x00 <VendorUnique> }
}
          \end{verbatim}
        \end{listing}

\end{itemize}

\paragraph{Errors}
After a Request Sense SCSI command, the reply will be scanned for known
errors, which are defined with the following error description block:

\index{error@{\it error}}
\index{code@{\it code}}
\index{condition@{\it condition}}

\begin{listing}
  \begin{verbatim}
error
{
  code        0x47110815  # This value is returned as the Helios error
                          # code if all of the following conditions are
                          # met.

  condition               # A condition description block, see below. Any
  {                       # number of these may appear.
    :
  }

  :

  condition
  {
    :
  }

}
  \end{verbatim}
\end{listing}

\subparagraph{Conditions}
Conditions are evaluated in order of appearance. If all conditions for an
error code are fulfilled, the associated error code will be returned.

\index{condition@{\it condition}}
\index{offset@{\it offset}}
\index{mask@{\it mask}}
\index{value@{\it value}}

\begin{listing}
  \begin{verbatim}
condition
{
  offset  0x08  # Byte offset into the Request Sense reply data.
  mask    0xFF  # Bit mask to apply to the addressed byte.
  value   0x15  # Value to be expected after masking out unused bits.
                # If this value is found, the condition is satisfied.
}
  \end{verbatim}
\end{listing}

\paragraph{Requests}
These keywords are used to specify a SCSI command sequence:

\index{request@{\it request}}
\index{fncode@{\it fncode}}
\index{item@{\it item}}

\begin{listing}
  \begin{verbatim}
request
{
  fncode 0x06060842  # Helios function code, for which the sequence shall
                     # be submitted.

  item  Mode_Select  # One of the earlier defined commands, which shall be
                     # executed.
}
  \end{verbatim}
\end{listing}

\subsubsection{Compiling}
If this source file finally meets your environment, compile it to binary form
with:

\medskip
\fbox{\tt \% gsi scsiinfo.src /helios/etc/scsiinfo}\index{gsi@{\tt gsi}}
\medskip

\subsection{Testing \SI/\DI\ entries}\label{sec:testinfo}
After recompiling both the \SI\ file and the \DI\ file, you should ensure that
the driver will work correctly with the new configuration. Another test
utility named {\tt testinfo}\index{testinfo@{\tt testinfo}} allows you to
simulate Helios requests as if they were comming from the file server.
{\tt testdrv} doesn't support multivolume\slash multipartition \DI s. Only
volumes that consist of one partition will be tested correctly.

\begin{caution}
  The {\tt testinfo} utility does {\bf not} preserve any data previously
  contained in the tested drives. Prove that you have a complete set of backups
  for all tested drives.
\end{caution}

Starting {\tt testinfo} is as follows:

\medskip
\fbox{\tt \% remote MSC testinfo msc21}
\medskip

\begin{screendump}
  \begin{verbatim}
   Defined Volumes : 3
           0 :     tandberg
           1 :     exabyte
           2 :     wren

   (O)pen Volume  open (A)ll  (Q)uit :
  \end{verbatim}
\end{screendump}


The program starts and displays the list of volumes as defined in \DI\
(imagine that each volume occupies a complete drive). Now you may select a
command by pressing one of the keys enclosed in brackets. When a volume has
been opened successfully, the program creates a new window for that volume,
using the volume name for the window name. After opening e g the Tandberg tape
streamer, the main window will look as follows:

\begin{screendump}
  \begin{verbatim}
   Defined Volumes : 3
           0 :     tandberg
           1 :     exabyte
           2 :     wren

   (O)pen Volume  open (A)ll  (Q)uit : o

   Which Volume : 0

   sending Open request for Volume #0 ( tandberg )
   time = 36 ms, Actual 0, Result 0x0

   Load result :   Error 0x0

           Raw 1, Removable 1, Loaded 1
           Protected 0, Formatted -1, NotLocked 0

   Defined Volumes : 3
           0 : *   tandberg
           1 :     exabyte
           2 :     wren

   (O)pen Volume  open (A)ll :
  \end{verbatim}
\end{screendump}


In this example, the {\tt FG\_Open} request was completed successfully within
36 ms. The Load result gives further information about the drive's state and
type: The drive is of type raw and uses removable media. The medium which has
been loaded is not write-protected, but no access check has been made. In the
volume list the tandberg drive is marked open with the asterisk. Furthermore,
{\tt (Q)uit} is disabled until all volumes are closed. There are two types of
volume windows, for disc drives and for tape drives, reflecting the different
command sets for each drive type. For the Tandberg streamer, a tape window is
opened and a menu offers several requests to execute and a test to perform.
We choose to perform the test and are asked for the {\tt File Size} and the
{\tt Files per Loop}. After typing in `{\tt 20}' and `{\tt 8}', we make the
tape to rewind first, and the test begins. After a few minutes, we stop the
test by pressing a key, the tape is rewound a short summary is printed to the
screen.

\begin{screendump}
  \begin{verbatim}
                                                                        tandberg

HandleTape #0 (tandberg) :
(G)etSize  (R)ead  (W)rite  (S)eek  (E)rase  (F)ilemark  (T)est  (C)lose : t
File Size ( 4 MByte ) : 20
Files per Loop ( 8 ) :
1: Space reverse, test whole tape  0: Rewind, test from start : 0
Testting Tape tandberg: From start with 8 files of 20 MByte, loop 1:
Rewind...                      53 ms
Write Filemark...             687 ms, file avg 77 KByte/sec
Write   501...                643 ms,   99 KByte/sec, bus avg  103 KByte/sec
Test stopped.
Rewind...                   38577 ms
Maximum value after 1 Loop:
Write count : 502, Reda Count 0, Tape Size 41024 units of 1024 bytes
Number of Requests with data errors : 0

HandleTape #0 (tandberg) :
(G)etSize  (R)ead  (W)rite  (S)eek  (E)rase  (F)ilemark  (T)est  (C)lose : c
Executing close request ...
time: 101 ms, Actual: 0, Result 0x0
press any key to close the window...
  \end{verbatim}
\end{screendump}


There are some more items in the menu to choose from, here is a short
description:

\begin{itemize}

  \item [`{\tt G}'] Perform a GetSize request. {\tt testinfo} reports the number
                  of blocks ans the block size.

  \item [`{\tt W}'] Perform a Write request. You have to enter the number of
                  units (whereas each unit contains \DI 's
                  {\it addressing} bytes) and the comparison seed. A sequence
                  of random number (initialized with the comparison seed) will
                  be written do tape. The time needed, the number of bytes
                  actally written and the Helios error code is reported. In
                  case of failure a Request Sense will be executed and its
                  result will appear, too.

  \item [`{\tt R}'] Perform a Read request. You have to enter the number of
                  units and the comparison seed, too. Based on that seed, the
                  same sequence of random numbers is generated again and is
                  expected to match the sequence written to tape before.
                  {\tt testinfo} reports time, bytes, error code and sense
                  data (if required).

  \item [`{\tt S}'] Perform a Seek request. You have to choose wether to look
                  for blocks, for file marks, or for sequential file marks.
                  The number of blocks/marks to skip has to be entered.

  \item [`{\tt E}'] Perform a Format request.

  \item [`{\tt F}'] Perform a Write Filemark request. Please enter the number
                  of file marks to write.

  \item [`{\tt T}'] Conduct a test like described above.

  \item [`{\tt C}'] Perform a Close request and close the window.

\end{itemize}

To finish this mini-session, press `{\tt c}', which causes {\tt testinfo} to
close the {\tt tandberg} window and to return to the main menu.

A test session with a random access drives is quite similar, an example could
be as follows:

\begin{screendump}
  \begin{verbatim}
                                                                            wren
executing GetSize request ...
Device has 647926 block of 1024 bytes.

HandleDisc #2 (wren) :
(R)ead  (W)rite  (F)ormat  (G)etSize  (T)est  (C)lose : t
Test disc wren :
Number of requests ( 16 ): 16
Request size ( 1 .. 64 units ) : 64
Allocted 16 requests of 64 units.
Test: (S)ame (C)ontiguous op(T)imised (R)andom (O)ptimised (E)nd : o
Press any key to stop :
Random Optimised :
wwwwwwwwwwwwwwww
16 requests of 64 KByte each written in 1790 ms ( 571 KBytes/sec )
rrrrrrrrrrrrrrrr
16 requests of 64 KByte each read in 1781 ms ( 574 KBytes/sec )
Random Optimised :
wwwwwwwwwwwwwwww
16 requests of 64 KByte each written in 1730 ms ( 591 KBytes/sec )
rrrrrrrrrrrrrrrr
16 requests of 64 KByte each read in 1731 ms ( 591 KBytes/sec )
Test: (S)ame (C)ontiguous op(T)imised (R)andom (O)ptimised (E)nd : e
  \end{verbatim}
\end{screendump}


In addition to the tape menu the disc menu offers some different kind of
tests. After entering the number of requests and the size of each request, you
have the choice between 5 tests:

\begin{itemize}

  \item [`{\tt S}'] Repeat the last test.

  \item [`{\tt C}'] Read/write some blocks that are in series on disc. Wait
                    for the completion of each request.

  \item [`{\tt T}'] Like `{\tt C}' , but its waited for the undevided
                    completion of the requests.

  \item [`{\tt R}'] Read/write some blocks placed in different disc areas.

  \item [`{\tt O}'] Like, `{\tt R}', but wait for undevided completion.

\end{itemize}
