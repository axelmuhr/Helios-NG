\appendix

\chapter{The configuration files}

\section{Quick reference to the \SI\ keywords}\index{scsiinfo@\SI}
The following section describes each keyword, the type of value which it
expects and the default value assumed if the keyword is omitted. The default
value {\tt <None>} marks keywords which must be given a value.

{\small
  \begin{tabular}{llrp{6.5cm}}
    keyword     & member of & default                & meaning \\
    \hline
    blockmove   & command   & no                     & Turn blockmove mode on/off \\
    cdb         & command   & {\tt <None>}           & Command Descriptor Block \\
    cdbsize     & command   & {\tt <GroupDependent>} & Size of the Command Descriptor Block in bytes \\
    code        & error     & {\tt <None>}           & Helios error code \\
    command     & device    & {\tt <None>}           & SCSI command description \\
    condition   & error     & {\tt <None>}           & Condition description \\
    data        & command   & {\tt <NoData>}         & Parameter list/replied data \\
    datasize    & command   & 0                      & Size of parameter list or replied data \\
    error       & device    & {\tt <None>}           & Error description \\
    fncode      & request   & {\tt <None>}           & Helios function code \\
    ident       & device    & {\tt <None>}           & Reply of the Inquiry SCSI command \\
    item        & request   & {\tt <NoCommand>}      & Command to be executed \\
    mask        & condition & {\tt <None>}           & Bit mask to apply to the addressed byte \\
    name        & command   & {\tt <None>}           & Command's name \\
    offset      & condition & {\tt <None>}           & Byte offset into the Request Sense reply data \\
    read        & command   & {\tt <None>}           & Direction of data flow \\
    request     & device    & {\tt <None>}           & Request description \\
    type        & device    & {\tt <None>}           & Selects disc or tape devices \\
    value       & condition & {\tt <None>}           & Expected value after masking out unused bits. \\
  \end{tabular}
}  

\section{\SI\ syntax definition}\label{app:si_syntax}
\index{scsiinfo@\SI}
The following syntax definition specifies all items and their order of
appearance, using these symbols:

\begin{tabular}{lp{12cm}}  
 [ ]\index{$[ ]$}           & Square brackets surround items that are
                              optional.      \\
\{ \}\index{$\lbrace \rbrace$}
                            & Braces surround optional items that can be
                              repeated.      \\
$< >$  \index{$< >$}        & Fishtail brackets enclose generic
                              expectations.  \\
\ldots \index{$\ldots$}     & Three dots denote continuation of a series; for
                              example, 12\ldots 89 is the same as the series
                              123456789.     \\
$|$    \index{$\vert$}      & A vertical bar separates two mutually exclusive
                              alternatives.  \\
\end{tabular}
  
{\small
\begin{tabular}{rp{11cm}}
   scsiinfo ::= & $<$entry$>$ \{�$<$entry$>$�\} \\
      entry ::= & `device' $<$name$>$ $<$description$>$ \\
description ::= & `\{' $<$type$>$ $<$ident$>$ \{�$<$command$>$�\} \{�$<$error$>$�\} \{�$<$request$>$�\} `\}' \\
       type ::= & `type' `sequential' $|$ `raw' $|$ `random' $|$ `structured' \\
      ident ::= & `ident' $<$name$>$ \\
    command ::= & `command' `\{' $<$cmdname$>$ $<$read$>$ $<$cdb$>$ [�$<$cdbsize$>$�] [�$<$datasize$>$�] [�$<$data$>$�] `\}' \\
    cmdname ::= & `name' $<$name$>$ \\
       read ::= & `read' `yes' $|$ `no' \\
        cdb ::= & `cdb' `\{' $<$number$>$ \{�$<$number$>$�\} `\}' \\
    cdbsize ::= & `cdbsize' $<$number$>$ \\
   datasize ::= & `datasize' $<$number$>$ \\
       data ::= & `data' `\{' $<$data\_item$>$ \{�$<$data\_item$>$�\} `\}' \\
 data\_item ::= & $<$number$>$ $|$ $<$string$>$ \\
      error ::= & `error' `\{' $<$errcode$>$ $<$condition$>$ \{�$<$condition$>$�\} `\}' \\
    errcode ::= & `code' $<$number$>$ \\
  condition ::= & `condition' `\{' $<$offset$>$ $<$mask$>$ $<$value$>$ `\}' \\
     offset ::= & `offset' $<$number$>$ \\
       mask ::= & `mask' $<$number$>$ \\
      value ::= & `value' $<$number$>$ \\
    request ::= & `request' `\{' $<$fncode$>$ \{�$<$item$>$�\} `\}' \\
     fncode ::= & `fncode' $<$number$>$ \\
       item ::= & `item' $<$name$>$ \\
       name ::= & $<$word$>$ $|$ $<$string$>$ \\
     string ::= & `''' \{�$<$any printable ascii character$>$�\} `''' \\
       word ::= & $<$letter$>$ \{�$<$letter$>$ $|$ $<$digit$>$ $|$ $<$underscore$>$�\} \\
     number ::= & [�`-'�] $<$ovalue$>$ $|$ $<$dvalue$>$ $|$ $<$xvalue$>$ \\
     ovalue ::= & `0' \{�$<$odigit$>$�\} \\
     dvalue ::= & $<$nzdigit$>$ \{�$<$digit$>$�\} \\
     xvalue ::= & `0' `x' $|$ `X' $<$xdigit$>$ \{�$<$xdigit$>$�\} \\
     odigit ::= & `0' \ldots `7' \\
      digit ::= & `0' \ldots `9' \\
    nzdigit ::= & `1' \ldots `9' \\
     xdigit ::= & `0' \ldots `9' $|$ `A' \ldots `F' $|$ `a' \ldots `f' \\
     letter ::= & `A' \ldots `Z' $|$ `a' \ldots `z' \\
 underscore ::= & `\_' \\
\end{tabular}
}

\section{Quick reference to the \DI\ keywords}\index{devinfo@\DI}

There are some keywords that must be given a fixed value:

{\small
  \begin{tabular}{llrp{7.5cm}}
    keyword    & member of  &   must be & meaning \\
    \hline
    addressing & discdevice &      1024 & Discdevice's addressing size \\
    blocksize  & fileserver &      4096 & fileserver's blocksize (interpreted in bytes) \\
    hugepkt    & fileserver &         1 & Size of huge cache packets (interpreted in fileserver's blocksize) \\
    mediumpkt  & fileserver &         4 & Size of medium cache packets (interpreted in fileserver's blocksize) \\
    name       & discdevice & msc21.dev & Device driver's filename (placed in {\tt /helios/lib}) \\
    smallpkt   & fileserver &        16 & Size of small cache packets (interpreted in fileserver's blocksize) \\
  \end{tabular}
}

The other keyword's values a chosen by the user. A default of {\tt<None>} forces the
user to give a value to the keyword.

{\small
\begin{tabular}{llrp{6.5cm}}
keyword     & member of  &     {\tt gdi} default & meaning \\
\hline
controller  & discdevice &                     0 & SCSI-controller's address \\
cgsize      & volume     & {\tt  <BestPossible>} & Cylinder group's size
                                                   (interpreted in fileserver's
                                                   blocksize) \\
device      & fileserver &          {\tt <None>} & Discdevice the fileserver will
                                                   use \\
discdevice  &            &          {\tt <None>} & Start of a discdevice block \\
drive       & discdevice &          {\tt <None>} & Start of a drive block \\
            & partition  &                     0 & Drive where the partition takes
                                                   place (interpreted in a drive
                                                   list's number that's generated by
                                                   the file server) \\
end         & partition  &     {\tt <LastBlock>} & Last block that is claimed by the
                                                   partition (interpreted in
                                                   discdevice's addressing size) \\
fileserver  &            &          {\tt <None>} & Start a file server block \\
hugecount   & fileserver &     {\tt <Tolerable>} & Number of huge packets in the
                                                   cache \\
id          & drive      &                     0 & Drive's SCSI-address \\
mediumcount & fileserver &     {\tt <Tolerable>} & Number of huge packets in the cache \\
minfree     & volume     &                     0 & Space to be left free for the checker's {\tt /lost+found} directory (interpreted in fileserver's blocksize) \\
name        & volume     &          {\tt <None>} & Volume's name \\
ncg         & volume     &  {\tt <BestPossible>} & Number of cylinder groups of volume \\
partition   & discdevice &          {\tt <None>} & Start of a partition block \\
            & volume     &          {\tt <None>} & Partition that's occupied by volume (interpreted in a partition list'snumber that's generated by the file server) \\
smallcount  & fileserver &     {\tt <Tolerable>} & Number of small packets in the cache \\
start       & partition  &                     0 & First block that is claimed by the partiton (interpreted in discdevice's addressing size) \\
syncop      & fileserver &                     0 & Switch to toggle between partly (= 0) or full (= 1) synchronous mode \\
type        & drive      &          {\tt <None>} & Drive's type (refering to \SI\ entires) \\
            & volume     &            structured & Volume's type \\
volume      & fileserver &          {\tt <None>} & Start of a volume block \\
\end{tabular}
}

\chapter{Errors}
There are several error messages that may occur when starting\slash running
the file server or any of the utilitiy commands. We differ between fatal error
conditions that lead to stopping a program (e g not enough memory for the
file server and its buffer cache) and normal errors\slash warnings that result 
in non performing of command (e g a full disc) but keep the file server running.
Most of the messages explain theirselves, here are the most important ones:

\begin{tabular}{lp{10cm}}
  Text:                & {\tt Failed to load devinfo.}                      \\
  Description:         & The file server did not find the device information
                         file that usually has to be placed in \HE\ or given as
                         parameter.                                         \\
  Corrective measures: & Create the required file by editing the \HEDIS\ file 
                         and compiling that via the {\tt gdi} command.      \\
\end{tabular}

\begin{tabular}{lp{10cm}}
  Text:                & {\tt Failed to find filesystem info.}               \\
  Description:         & The value of the last parameter of {\tt fs} and the
                         value of the {\it fileserver} keyword in your \DI\
                         are not equal.                                      \\
  Corrective measures: & Reenter your {\tt fs} call (if there was an error in
                         typing) or correct the value in your \DI\ (don't
                         forget to use the {\tt gdi} command afterwards).\\
\end{tabular}

\begin{tabular}{lp{10cm}}
  Text                 & {\tt Failed to find discdevice info.}                \\
  Description:         & In your \DI\ there's no discdevice block
                         corresponding to the value you gave to {\it device}
                         in the {\it fileserver} block.                                               \\
  Corrective measures: & Complete your \DI\ and recompile it (use {\tt gdi}). \\
\end{tabular}

\begin{tabular}{lp{10cm}}
  Text                 & {\tt Failed to open device.}                       \\
  Description:         & The {\tt OpenDevice()} call couldn't succeed.      \\
  Corrective measures: & \begin{itemize}

                           \item Usually, the device driver's name as written
                                 in \DI's discdevice block (following the name
                                 keyword) does not correspond to an existing
                                 file.

                                 \begin{note}
                                   The name should be {\tt msc21.dev} and is
                                   expanded to
                                   {\tt /helios\slash lib\slash msc21.dev}.
                                   Assert that this file contains your device
                                   driver.
                                 \end{note}

                           \item Maybe you didn't start the file server on a
                                 MSC node. The correct call is

                                 \fbox{\tt \% remote -d MSC fs msc21}

                           \item You have an old MSC board in use and didn't 
                                 perform an update of your hardware. See section
                                 \ref{sec:hwupdate} for further information.

                           \item Else there could be problems with your SCSI
                                 devices or the SCSI bus. Check your hardware
                                 and try it again.

                         \end{itemize}
\end{tabular}

\begin{tabular}{lp{10cm}}
  Text                 & {\tt Failed to to init device info for fileserver.} \\
  Description:         & Have a look at the server window to get detailed
                         information.                                        \\
  Corrective measures: & Correct the errors by editing your \DI\ or clearing
                         the memory\\
\end{tabular}

\begin{tabular}{lp{10cm}}
  Text:                & {\tt Not enough memory for PFS v2.1.}               \\
                       & or                                                  \\
                       & {\tt Failed to allocate memory for server.}         \\
  Description:         & There was not enough memory for the file server's
                         buffer cache etc.                                   \\ 
  Corrective measures: & \begin{itemize}

                           \item Have a look at your \DI\ file. Maybe your 
                                 buffer cache appotionment via
                                 {\it smallcount}\slash
                                 {\it medium\-count}\slash {\it huge\-count}
                                 results in asking for more memory than the
                                 MSC board offers. Correct the values.
                           \item You've already started a file server and some 
                                 difficulties occured, so that you wanted to 
                                 restart the file server, but it's still in 
                                 memory. Note that you have to terminate 
                                 {\bf all} volumes, even though one of them was
                                 not loaded.

                         \end{itemize}                                       \\
\end{tabular}
