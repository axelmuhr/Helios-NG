\chapter{Commands} \label{chap:commands}

The following commands can be executed from the command line and are supplied to
make better use of the Parsytec File System.

\section{access}\index{acess@{\tt access}}
\begin{man}
  \PP Reports access rights of a file.
  \FO {\tt access <File>}
  \DE {\tt access} reports the user's access rights of file. For more details
      see ``The Helios Parallel Operating System'', section 3.4,
      ``Protection: a tutorial''.
  \SA {\tt chmod}, {\tt matrix}, {\tt refine}
\end{man}

\section{chmod}\index{chmod@{\tt chmod}}
\begin{man}
  \PP Alter the protection bits of a file.
  \FO {\tt chmod [vxyz][+-=] [rwefghvxyzda] <File>}
  \DE {\tt chmod} is used to enable and disable the protection bits of a
             {\tt <File>}. For more details see ``The Helios Parallel Operating
             System'', section 3.4, ``Protection: a tutorial''.
  \SA {\tt access}, {\tt matrix}, {\tt refine}
\end{man}

\section{finddrv}\index{finddrv@{\tt finddrv}}
\begin{man}
  \PP Looks for SCSI devices.
  \FO {\tt finddrv <DiscDevice>}
  \DE {\tt finddrv} reads \HEDI\ and loads the driver specified
             in the {\tt <DiscDevice>} block. The driver reads the
             \HESI\ file and tests every SCSI address for devices.
\end{man}

\section{format}\index{format@{\tt format}}
\begin{man}
  \PP Format a disc.
  \FO {\tt format <PathToVolume>}
  \DE This command works with optical discs only.
             The specified volume is tried to be formatted physically. After
             formatting each sector is verified and reassigned on error. The
             volume has to be loaded with the {\tt -m} option otherwise {\tt format}
             returns an error. Depending on medium size the execution of this
             command may last some time because the {\tt format} command scans every
             physical sector on disc.
  \SA {\tt makefs}, {\tt mksuper}
\end{man}

\section{fs}\index{fs@{\tt fs}}
\begin{man}
  \PP Start the Parsytec File Server.
  \FO {\tt fs [-c][-o][-f|-b|-n] [<DevInfoName>] <FileServer>}
  \DE The file server is started, loads {\tt <DevInfoName>} (when given), tries
             to install the file server specified in the {\tt <FileServer>} (usually of
             value `{\tt msc21}') block. The file server has to be started on a MSC in the
             background. This is normally done with

             \bigskip
             \fbox{\tt remote -d MSC fs msc21}
             \bigskip

             Options:
             \begin{itemize}

             \item {\tt -c}\newline
               The use of the buffer cache checksum is enabled. Using this mode
               reduces speed of file server by factor three.
	       Default is working without buffer cache checksum.

             \item {\tt -o}\newline
               If this option is given the file server reports open requests
               to the server window.

             \item {\tt -f}, {\tt -b}, {\tt -n}\newline
               These options work with structured volumes only.
	       The checking mode is determined. This option is overridden by 
               the checking mode option of a {\tt load} command. Default checking
               mode is {\tt -f}.

               \begin{itemize}

               \item {\tt -f}\newline
                 Full checks; file system data and directory trees are checked.

               \item {\tt -b}\newline
                 Basic checks; file system data is checked and on occurrence of
	         errors directory trees are checked.

               \item {\tt -n}\newline
                 No checks; checker is bypassed completely.

               \end{itemize}
             \end{itemize}

             \begin{note}
               {\tt fs} allocates memory for all volumes specified in \DI. So you
               have to execute a {\tt termvol} command for those volumes to terminate
               the file server and to clean the memory, even if some volumes have
               not been loaded.
             \end{note}

  \SA {\tt load}, {\tt unload}, {\tt termvol}
\end{man}

\section{fsync}\index{fsync@{\tt fsync}}
\begin{man}
  \PP Toggle between partly and fully synchronous mode.
  \FO {\tt fsync <PathToVolume> [-as]}
  \DE {\tt fsync} allows the selection between two operation modes: at volume
             load time the default mode is the partly synchronous mode ({\tt -a})
             which means that all data-blocks are written with a certain delay
             (of max. 20 seconds) to disc, when the ``sync process''---which is
             part of the server---becomes active and detects some of them. To
             guarantee that all blocks are written directly to disc
             (``write-through-cache''), the user has the alternative to switch to
             fully synchronous mode ({\tt -s}) , which eliminates all delayed-write
             operations.
  \SA {\tt sync}
\end{man}

\section{gdi}\index{gdi@{\tt gdi}}
\begin{man}
  \PP Generate a ``device information file''.
  \FO {\tt gdi <Input> <Output>}
  \DE {\tt gdi} is a simple compiler which generates a binary object from the
             given device information {\tt <Input>} file. The default filename which
             is searched by the server is \HEDI, so {\tt gdi}'s
             {\tt <Output>} should be that file. When using other values for
             {\tt <Output>} (necessary if you are using several MSCs with different
             \DI s), the file server has to be given the correct \DI\ as
             parameter.
  \SA {\tt gsi}
\end{man}

\section{gsi}\index{gsi@{\tt gsi}}
\begin{man}
  \PP Generates a ``SCSI-information file''.
  \FO {\tt gsi <Input> <Output>}
  \DE {\tt gsi} is a simple compiler which generates a binary object from the
             given SCSI-information {\tt <Input>} file. The file name which is
             searched by the MSC device driver is \HESI,
             so {\tt gsi}'s {\tt <Output>} has to be that file.
  \SA {\tt gdi}
\end{man}

\section{load}\index{load@{\tt load}}
\begin{man}
  \PP Load a volume.
  \FO {\tt load [-v][-l][-m][-f|-b|-n] <PathToVolume>}
  \DE The volume specified by {\tt <PathToVolume>} is loaded (mounted)
             and for structured volumes the checker is called and a file system
             is tried to be set up.
             After file server startup this has to be done with every volume
             explicitly.
             Changeable media are locked after they have been loaded so that
             they cannot be removed until an unload command is given.

             Options:
             \begin{itemize}

             \item {\tt -v}\newline
               The {\tt load} command waits for the completion of the load and
               reports about the results. On success {\tt load} reports about the
               number of cylinder groups and blocks per cylinder group of the
               loaded file system. This option has no effect if the {\tt -m} option
               is given. Default is not to wait for completion.

             \item {\tt -m}\newline
               This option works with structured volumes only.
	       The specific volume is loaded but the checker is not called and
	       no file system is tried to be set up. This option must be applied
	       before using the {\tt makefs}, {\tt format} or {\tt mksuper} command.
	       This option disables the {\tt -v} option.

             \item {\tt -f}, {\tt -b}, {\tt -n}\newline
               These options work with structured volumes only.
	       The checking mode is determined. If none of these options is
	       given the checking mode determined in the file server commandline
	       is used. If no specific checking mode was given there the 
               default mode ({\tt -f}) is used.

               \begin{itemize}
               \item {\tt -f}\newline
                 Full checks; file system data and directory trees are checked.

               \item {\tt -b}\newline
                 Basic checks; file system data is checked and on occurrence of
	         errors directory trees are checked.

               \item {\tt -n}\newline
                 No checks; checker is bypassed completely.
               \end{itemize}

             \item {\tt -l}\newline
               This option only has an effect during a full check.
	       If there are `hanging' symbolic links detected after a full
               check these links will be destroyed. Default is not to destroy
               `hanging' links.
             \end{itemize}

  \SA {\tt unload}, {\tt termvol}
\end{man}

\section{makefs}\index{makefs@{\tt makefs}}
\begin{man}
  \PP Create a file system.
  \FO {\tt makefs <PathToVolume>}
  \DE This command works with structured volumes only.
             A new file system is tried to be created on the volume depending
             on the volume\slash partition description in \HEDI\ (that was
             created via the {\tt gdi} command). The volume has to be loaded with
             the {\tt -m} option and physically formatted, otherwise {\tt makefs} returns
             an error. On success {\tt makefs} reports about the number of cylinder
             groups and blocks per cylinder group of the file system it has
             installed.

             \begin{caution}
               When accidently running on a data containing file system, all
               files will be destroyed.
             \end{caution}
  \SA {\tt format}, {\tt load}, {\tt mksuper}
\end{man}

\section{man}\index{man@{\tt man}}
\begin{man}
  \PP Prints command description.
  \FO {\tt man <PFScommand>}
  \DE Prints (via {\tt more}) a short description of the given {\tt <PFScommand>}.
\end{man}

\section{matrix}\index{matrix@{\tt matrix}}
\begin{man}
  \PP Display the access matrix of a file.
  \FO {\tt matrix <File>}
  \DE The utility {\tt matrix} displays the access matrix of the given {\tt <File>}.
             For more details see ``The Helios Parallel Operating System'',
             section 3.4, ``Protection: a tutorial''.
  \SA {\tt access}, {\tt chmod}, {\tt refine}
\end{man}

\section{mksuper}\index{mksuper@{\tt mksuper}}
\begin{man}
  \PP Generate a superblock.
  \FO {\tt mksuper <PathToVolume>}
  \DE This command works with structured volumes only.
             A superblock (info block 0) of the specified volume is constructed
             depending on the \HEDI\ information and written to
             disc.
             The volume has to be loaded with the {\tt -m} option, otherwise
             mksuper will return an error.
             This command should only be used if the checker failed because of
             a corrupted superblock.
  \SA {\tt format}, {\tt load}, {\tt makefs}
\end{man}

\section{ptar}\index{ptar@{\tt ptar}}
\begin{man}
  \PP Store files in an archive.
  \FO {\tt ptar <Options> <Files>}
  \DE {\tt ptar} allows you storing copies of files in an archive.

             Options:

             \begin{itemize}
             \item {\tt -c}, {\tt -d}, {\tt -t}, {\tt -x}\newline
               These four option switch between the operation modes.

               \begin{itemize}
               \item {\tt -c}\newline
                 Create a new archive.

               \item {\tt -d}\newline
                 Compare the files in the archive with those in the file system
                 and report about differences.

               \item {\tt -t}\newline
                 Display a list a the files in the archive.

               \item {\tt -x}\newline
                 Extract files from archive.
               \end{itemize}

             \item {\tt -B}, {\tt -C}, {\tt -f}, {\tt -M}, {\tt -N}, {\tt -R}, {\tt -T}, {\tt -v}, {\tt -w}\newline
               General options.

               \begin{itemize}
               \item {\tt -B <Number>}\newline
                 Set blocking factor to {\tt <Number>}.

               \item {\tt -C <Directory>}\newline
                 Change into {\tt <Dirtectory>} before continuing.

               \item {\tt -f <Filename>}\newline
                 Archive files in {\tt <Filename>} (instead of using the value of
                 {\tt TARFILE} respectively `{\tt tar.out}').

               \item {\tt -M}\newline
                 Work on a multi-volume archive.

               \item {\tt -N <Date>}\newline
                 Work only on files whose creation or modification date is
                 newer than {\tt <Date>}.

               \item {\tt -R}\newline
                 Print each message's record number.

               \item {\tt -T <Filename>}\newline
                 Work on the list of files in {\tt <Filename>}, too.

               \item {\tt -v}\newline
                 Enter verbose mode.

               \item {\tt -w}\newline
                 Wait for user's confirmation before every action.
               \end{itemize}
             \end{itemize}
\end{man}
             \begin{itemize}
             \item {\tt -h}, {\tt -V}\newline
               Creation options.

               \begin{itemize}
               \item {\tt -h}\newline
                 Treat simbolic links as normal files or directories.

               \item {\tt -V <Name>}\newline
                 Write a volume header at the beginning of the archive.
               \end{itemize}

             \item {\tt -k}, {\tt -m}, {\tt -p}\newline
               Extraction option:

               \begin{itemize}
               \item {\tt -k}\newline
                 Keep existing files in the file system.

               \item {\tt -m}\newline
                 Do not extract the modification and access date from archive.

               \item {\tt -p}\newline
                 Set access matrices as recorded in the archive.
               \end{itemize}
             \end{itemize}

             See chapter ``Backups'' for detailed information.
\newpage

\section{refine}\index{refine@{\tt refine}}
\begin{man}
  \PP Refine or restrict a capability.
  \FO {\tt refine [-+=][rwefghvxyzda] <File>}
  \DE refine allows refining and restricting of capabilities associated
             to a {\tt <File>}. For more details see ``The Helios Parallel
             Operating System'', section 3.4, ``Protection: a tutorial''.
  \SA {\tt access}, {\tt chmod}, {\tt matrix}
\end{man}

\section{sync}\index{sync@{\tt sync}}
\begin{man}
  \PP Force a sync operation immediately.
  \FO {\tt sync <PathToVolume>}
  \DE The utility {\tt sync} forces an ``extra'' sync operation which guarantees
             that all data blocks in the buffer cache with the ``delayed-write''
             flag set are written immediately to disc. {\tt sync} is especially
             useful to guarantee consistency if the file server or the whole
             system shall be shut down.
  \SA {\tt fsync}
\end{man}

\section{termvol}\index{termvol@{\tt termvol}}
\begin{man}
  \PP Terminate a volume.
  \FO {\tt termvol [-v] <PathToVolume>}
  \DE The volume specified by {\tt <PathToVolume>} is terminated.
             Therefore the volume is unloaded and then the volume specific
             central server is terminated. A volume which has been terminated 
             cannot be loaded again before starting the file server again. If 
             all volumes have been terminated the whole file server will 
             terminate automatically.
             \begin{note}
               With respect to safety of data there's no command to terminate
               all volumes (and thereby the complete file server) at once.
             \end{note}

             Option:

             \begin{itemize}
             \item {\tt -v}\newline
               The termvol command waits for the completion of the volume
               termination.
             \end{itemize}

  \SA {\tt fs}, {\tt load}, {\tt unload}
\end{man}

\section{testdrv}\index{testdrv@{\tt testdrv}}
\begin{man}
  \PP Perform low-level SCSI commands.
  \FO {\tt testdrv <DiscDevice>}
  \DE {\tt testdrv} reads \HEDI\ and loads the driver specified
             in the {\tt <DiscDevice>} block. It lets you compose SCSI requests and 
             performs them on a specified drive. {\tt testdrv} must be executed on the 
             transputer placed on the MSC board, so you must type e g

             \bigskip
             \fbox{\tt \% remote MSC testdrv msc21}
             \bigskip

             \begin{caution}
               Data is {\bf not} preserved, you have to backup the drives that are
               intended to be tested.
             \end{caution}
  \SA Section \ref{sec:testdrv}
\end{man}

\section{testinfo}\index{testinfo@{\tt testinfo}}
\begin{man}
  \PP Test devices described in \DI\slash \SI.
  \FO {\tt testinfo <FileServer>}
  \DE {\tt testinfo} performs some menu controlled standard test with the
             drives described in the information files (whereas multivolume\slash
             multipartition \DI s are not supported). {\tt testinfo} must be
             executed on the transputer placed on the MSC board, so you must
             type e g

             \bigskip
             \fbox{\tt \% remote MSC testinfo msc21}
             \bigskip

             \begin{caution}
               Data are {\bf not} preserved, you have to backup the drives that are
               intended to be tested.
             \end{caution}
  \SA Section \ref{sec:testinfo}
\end{man}

\section{unload}\index{unload@{\tt unload}}
\begin{man}
  \PP Unload a volume.
  \FO {\tt unload [-v] <PathToVolume>}
  \DE The volume specified by {\tt <PathToVolume>} is unloaded
             (unmounted). Actually working processes on that volume are closed
             and the volume is updated. Protected media are unlocked so that
             they can be removed. In contrary to the {\tt termvol} command the
             central server for this volume is not terminated so that the 
             volume may be loaded again.

             Option:

             \begin{itemize}
             \item {\tt -v}\newline
               {\tt unload} waits for the completion of the volume unloading.
             \end{itemize}

  \SA {\tt fs}, {\tt load}, {\tt termvol}
\end{man}
